\documentclass[11pt,twoside,a4paper]{article}

\usepackage{mathtools}
\usepackage{amsmath}

\setlength{\parindent}{0pt}

\begin{document}

Main question what does effect the randomization of the error??
	
\begin{itemize}
\item Motors do not consume the same amount of Energy between each checkpoint due to non-constant Power consumption. ( Due to variating torque applied to the motor)?
\item If a timer is used than checkpoint not always in the same spot, this may also vary the Energy consumed (Probably only if a lot of computation is done for a single checkpoint)?
\item Delay between issuing and actually doing the action, same holds for stopping
\end{itemize}


Assumptions: \ 
\begin{itemize}
\item The duration of the actuation event is assumed to not fit in a single energy cycle
\item The Power used during actuation is not constant, sum of computation and actuation.
\item Start with a full energy buffer
\item During the active period the energy in the storage device is not increased (device not charged)
\item The energy storage device is perfect in terms of leakage
%
%\item Pact \textgreater\textgreater\phantom Pcomp therefore Pcomp is neglectible (while check pointing takes a finite amount of time which is dominated by the actuation event)

\end{itemize}


Variables: \\
\begin{itemize}
\item The length of the total actuation event is equal to L.
\item The harvested energy is stored in a storage device (Super capacitor), with capacity C.
\item The energy used by actuation step between two checkpoints is equal to E.
\item Check pointing is done with a fixed interval n
\item The amount of power consumed in the time steps t, are independent and normally distributed with an average power consumption $\mu_p$ and a standard deviation $\sigma_p$.
\end{itemize}

\section{Model:}


Assume $t_{charge} > t_{discharge}$ (otherwise no duty cycling): 

\begin{equation}
E_{charged} = \int_{0}^{t_{\text{charged}}} P_{\text{in}}(t) \mathrm{d}t + \int_{t_{\text{charged}}}^{t_{\text{discharged}}} P_{\text{in}}(t)  \mathrm{d}t
\end{equation}

%If the storage element is a capacitor:

%\begin{equation}
%E_{charged} = \int_{0}^{t_{\text{charged}}} P_{\text{in}}(t) \mathrm{d}t + \int_{t_{\text{charged}}}^{t_{\text{discharged}}} %P_{\text{in}}(t)  \mathrm{d}t
%\end{equation}

%\begin{equation}
%E_{charged} = \frac{1}{2}C(V_{\text{max}}^2 - V_{\text{min}}^2) + \int_{t_{\text{discharged}}}^{t_{\text{charged}}} P_{\text{in}}(t)  \mathrm{d}t
%\end{equation}

\begin{equation}
E_{load} = \int_{t_\text{charged}}^{t_{\text{discharged}}} P_{\text{load}}(t) \mathrm{d}t 
\end{equation}

	
Assumed constant power used during actuation:

\begin{equation}
E_{load} = P_{\text{load}}(t_{\text{discharged}} - t_{\text{charged}})
\end{equation}

%\begin{equation}
%t_{\text{int}} = t_{\text{charged}} + t_{\text{discharged}}
%\end{equation}

%\begin{equation}
%D = \frac{t_{\text{discharge}}}{{t_{\text{charge}} + t_\text{discharge}}} 
%\end{equation}

%Assume a set of checkpoints N with n $\in$ N \\

%\begin{equation}
%t_{\text{cnt}}(n) = t_{\text{charged}} + n\cdot{}t_{\text{cnt period}}
%\end{equation}

\begin{equation}
n = floor(\frac{t_{\text{int}} - t_{\text{discharge}}}{t_{\text{discharge}}})
\end{equation}

\begin{equation}
\epsilon_{\text{cnt}} =
    \begin{cases}
      (n+1)\cdot{}t_{\text{cnt period}} - t_{\text{discharge}}, & \text{if } \text{counting before} \\
      t_{\text{discharge}} - n\cdot{}t_{\text{cnt\_period}}, & \text{if } \text{counting after}
    \end{cases}
\end{equation}

\newpage

\section{Model motor error}

Assumptions: \\
\begin{itemize}
\item Assume 1 motor or the two motors behave equal
\item Stepper motor high speed $->$ low torque, applied force only just bigger than friction force.
\item Denergizing the motor will bring the robot to a almost immediate stop.
\item Assume counting before doing the action.
\item Error in physical steps not doing the action but counted = 3
\item After every power interrupt the error is equal to 3
\end{itemize}


Model should also capture the relation between: \\
\begin{itemize}
\item Number of interrupts
\item Speed
\item Error, difference between software count and movement in one step due to pi
\end{itemize}

\end{document}