\documentclass[11pt,twoside,a4paper]{article}

\usepackage{mathtools}
\usepackage{amsmath}

\setlength{\parindent}{0pt}

\begin{document}

%The actuation is modeled as a continuous operation in time C = {P, I} which consists of a set of periods P = {1,...n} and a randomly distributed set of power interrupts I.
%At the beginning of each period there is a checkpoint, where you assume that the next period will be finished 



Assumptions: \ 
\begin{itemize}
\item Constant power usage during actuation
\item Pact \textgreater\textgreater\phantom Pcomp therefore Pcomp is neglectible
\item Start with a full energy buffer
\item Variating input power ie. incomming energy during actuation
%\item 
\end{itemize}


Variables: \\
 % / periods \\
%M = total number of power interrupts \\
%m $\in$ M \\
%length period = $ t_{p} = t' - t / N $

\section{Model:}


Assume $t_{charge} > t_{discharge}$ (otherwise no duty cycling): 

\begin{equation}
E_{charged} = \int_{0}^{t_{\text{charged}}} P_{\text{in}}(t) \mathrm{d}t + \int_{t_{\text{charged}}}^{t_{\text{discharged}}} P_{\text{in}}(t)  \mathrm{d}t
\end{equation}

%If the storage element is a capacitor:

%\begin{equation}
%E_{charged} = \int_{0}^{t_{\text{charged}}} P_{\text{in}}(t) \mathrm{d}t + \int_{t_{\text{charged}}}^{t_{\text{discharged}}} %P_{\text{in}}(t)  \mathrm{d}t
%\end{equation}

%\begin{equation}
%E_{charged} = \frac{1}{2}C(V_{\text{max}}^2 - V_{\text{min}}^2) + \int_{t_{\text{discharged}}}^{t_{\text{charged}}} P_{\text{in}}(t)  \mathrm{d}t
%\end{equation}

\begin{equation}
E_{load} = \int_{t_\text{charged}}^{t_{\text{discharged}}} P_{\text{load}}(t) \mathrm{d}t 
\end{equation}

	
Assumed constant power used during actuation:

\begin{equation}
E_{load} = P_{\text{load}}(t_{\text{discharged}} - t_{\text{charged}})
\end{equation}

\begin{equation}
t_{\text{int}} = t_{\text{charged}} + t_{\text{discharged}}
\end{equation}

%\begin{equation}
%D = \frac{t_{\text{discharge}}}{{t_{\text{charge}} + t_\text{discharge}}} 
%\end{equation}

Assume a set of checkpoints N with n $\in$ N \\

\begin{equation}
t_{\text{cnt}}(n) = t_{\text{charged}} + n\cdot{}t_{\text{cnt period}}
\end{equation}

\begin{equation}
n = floor(\frac{t_{\text{int}} - t_{\text{discharge}}}{t_{\text{discharge}}})
\end{equation}

\begin{equation}
\epsilon_{\text{cnt}}(n) =
    \begin{cases}
      t_{\text{cnt}}(n+1) - t_{\text{int}}, & \text{if } \text{counting before} \\
      t_{\text{int}} - t_{\text{cnt}}(n), & \text{if } \text{counting after}
    \end{cases}
\end{equation}

\newpage

\section{Model motor error}

Assumptions: \\
\begin{itemize}
\item Assume 1 motor or the two motors behave equal
\item Stepper motor high speed $->$ low torque, applied force only just bigger than friction force.
\item Denergizing the motor will bring the robot to a almost immediate stop.
\item Assume counting before doing the action.
\item Error in physical steps not doing the action but counted = 3
\item After every power interrupt the error is equal to 3
\end{itemize}


Model should also capture the relation between: \\
\begin{itemize}
\item Number of interrupts
\item Speed
\item Error, difference between software count and movement in one step due to pi
\end{itemize}

\end{document}