\documentclass[11pt,twoside,a4paper]{article}

\usepackage{mathtools}
\usepackage{amsmath}

\begin{document}

%The actuation is modeled as a continuous operation in time C = {P, I} which consists of a set of periods P = {1,...n} and a randomly distributed set of power interrupts I.
%At the beginning of each period there is a checkpoint, where you assume that the next period will be finished 


\section{Assumptions:}
 
\begin{itemize}
\item Constant power usage during actuation
\item Pact \textgreater\textgreater\phantom Pcomp therefore Pcomp is neglectible
\item Start with a full energy buffer
\item Variating input power ie. incomming energy during actuation
%\item 
\end{itemize}

\section{Variables:}

P = period \\% = t' - t
N = number of checkpoints / periods \\
n $\in$ N \\
%M = total number of power interrupts \\
%m $\in$ M \\
length period = $ t_{p} = t' - t / N $

\section{Model:}

\begin{equation}
I_{\text{}} = t_{\text{x}} + \lambda e^{-\lambda y} \text{ with: y = random variable}
\end{equation}

\begin{equation}
\epsilon_{\text{p}}(n) =
    \begin{cases}
      t_{\text{p}} - I_{\text{}}, & \text{if } I \text{ in } P(n)  \\
      0, & \text{otherwise}
    \end{cases}
\end{equation}

\begin{equation}
\epsilon_{\text{avg}} = \frac{1}{n}\sum_{n=1}^{N} \epsilon_{\text{p}}(n) 
\end{equation}


\section{Model motor error}

Assumptions: \\
\begin{itemize}
\item Assume 1 motor or the two motors behave equal
\item Stepper motor high speed $->$ low torque, applied force only just bigger than friction force.
\item Denergizing the motor will bring the robot to a almost immediate stop.
\item Assume counting before doing the action.
\item Error in physical steps not doing the action but counted = 3
\item After every power interrupt the error is equal to 3
\end{itemize}

\noindent
Model should also capture the relation between: \\
\begin{itemize}
\item Number of interrupts
\item Speed
\item Error, difference between software count and movement in one step due to pi
\end{itemize}

\end{document}