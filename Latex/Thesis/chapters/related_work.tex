\chapter{Related Work}
\label{chp:related_work}

\section{Transiently-powered systems}

Fully programmable RFID platforms have been developed to exploring the combination of sensing, computation and communication, while allowing battery-less operation by harvesting RF energy~\cite{sample_transim_2008}.
The amount of energy collected from RF signals is very small and decreases with the distance of the device to the transmitter.
The harvested energy is typically stored in a capacitor, where larger capacitors can buffer more energy and smaller capacitors have the advantage of shorter charge times~\cite{gummerson_mobisys_2010}.
For longer, complex operations the energy budged needs to be evaluated carefully.
To store the energy an appropriate size storage capacitor needs to be selected~\cite{naderiparizi_rfid_2015}.

% - Energy harvesting

%Other sources available for exploration are often limited by the application. Secondly, most sources can be scarce or completely absent during prolonged time intervals of the day as well \cite{RN15}. 

% - Short intro into persistent framwork: checkpointing etc

\section{Small robotic platforms}

Low cost robotic platforms have been developed to tackle a variety of challenges anonymously.
Miniature robots can be used for inspection in difficult to reach places, operating like mobile sensing units.
Hardware modularity is a way to make the robot adapt its resources to different environments and sensing operations.
By separating out power, computation, motor control and sensing a verity of capabilities can be tested~\cite{sabelhaus_icra_2013, pickem_icra_2015, kim_iros_2016}.
Microrobots typically use infrared-based neighbor to neighbor distance sensing and communication~\cite{rubenstein_icra_2012, pickem_icra_2015, kim_iros_2016}.
While controlling a swarm or collective is mainly accomplished by means of active low power transceivers~\cite{sabelhaus_icra_2013, pickem_icra_2015, kim_iros_2016}. 

\section{Continuous operation} 
%Battery replenishment

Typically the operation time is extended by regularly checking the remaining energy in the battery and move to a recharging station before the robot runs out of energy~\cite{pickem_icra_2015, rubenstein_icra_2012}.
As an alternative to quickly recharging, the battery can also be swapped automatically when the robot moves into the docking station~\cite{kemal_mech_2015}.
Another work shows a robot which is able to swap it's primary battery using a six degree-of-freedom manipulator, used to grab the dead battery and plug it into a wireless recharging charging station \cite{zhang_conel_2013}.
Using direct wireless power to replace or supplement to a batteries energy is shown in~\cite{karpelson_icra_2014}, however the robot can only operate or recharge while remaining in close proximity to a transmitter. 
In these cases the robots are highly reliant on an infrastructure to allow for continuous autonomous operation.
This can be a severe constraint if the robot moves out of reach or needs to operate in a area where this infrastructure is not present. Persistent operation can also be achieved by harvesting renewable energy, particularly solar energy to complement to the robots internal energy source. To remove weight from the robot, in \cite{bruhwiler_iros_2015} the solar energy is used directly without any type of energy buffer. A drawback of this method is that the incoming solar energy should already greater or equal to the energy required for operation. This approach has been tested for basic locomotion and did not combine any form of sensing and control.

\section{Path planning based on energy availability}

To capture the optimal amount of solar energy along the way, a map of the expected solar power can be used to compute the optimal path. To distinguish sunny or shaded two methods are proposed in \cite{plonski_tranro_2016}, one being a simple data driven Gaussian Process and the other estimates the geometry of the environment as a latent variable.
Energy aware path planning is commonly used in combination with mission planning.
In \cite{kaplan_iros_2016}, an analysis of the solar radiation is used to generate a time-optimized motion plan and power schedule using a cascaded particle swarm optimization algorithm.
By combining maps of lighting and ground slope a solar-powered robot can be kept illuminated continuously. A connected component analysis is used to plan a optimal route on traversable slopes, as described by \cite{otten_icra_2015}.



% - Provide overview table robots smaller than 15*15cm

% For each of these cases you need to provide numbers: 
% level of autonomy (does the robot does all by itself or relies on external processing)
% does autonomy fall under 
% charging time

% Add missing "new" robots

\begin{table*}[t]
	\centering
	\tiny
	\begin{threeparttable}
		\caption{An comparison of small robotic platforms}
		\label{tab:1}
 		\begin{tabularx}{\textwidth}{l l X X X l l l} 
			\hline
 			Robot & Cost & Scalability & Sensors & Locomotion & Size [cm] & Weight [g] & Battery life \\ 
 			\hline
 			IPR & TBD & charge, program & gyro & wheel, 3cm/s & 4.0 & 15 & 1s\\
 			HAMR-VP\textsuperscript{1} \cite{bruhwiler_iros_2015} & NS & none & gyroscope, optical mouse & legged, 1cm/s & 4.4 & 2.3 & 3m \\
 			Roverables \cite{dementyev_uist_2016} & NS & charge & wheel, distance, optical encoders & wheel, ?? & 4.0 & ?? & 45m \\ 
 			Zooids \cite{legoc_uist_2016} & \$50 & ?? & position, touch & wheel, 50cm/s & 2.6 & 12 & 1-2h \\ 
 			mROBerTO \cite{kim_iros_2016} & \$60\textsuperscript{1} & program & light, range, gyro, camera, accel., compass, distance, bearing & motor shaft, 15cm/s & 1.5 & ?? & 1.5h\\
 			GRITSBot \cite{pickem_icra_2015} & \$50\textsuperscript{2} & charge, program, calibrate & distance, bearing, 3d accel., 3d gyro & wheel, 25cm/s & 3 & ?? & 1-5h \\
 			Kilobot \cite{rubenstein_icra_2012} & \$50\textsuperscript{2} & charge, program & distance, ambient light & vibration, 1cm/s & 3.3 & ?? & 3-24h\\
 			TinyTerp \cite{sabelhaus_icra_2013} & \$50 & none & 3d gyro, 3d accel. & wheel, 50cm/s & 1.8 & ?? & 1h\\
			\hline
		\end{tabularx}
		\begin{tablenotes}
			\item [1] Modified to include on-board power, sensing and control.
			\item [2] Cost of parts
		\end{tablenotes}
	\end{threeparttable}
\end{table*}