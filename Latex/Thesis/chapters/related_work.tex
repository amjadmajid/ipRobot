\chapter{Related Work}
\label{chp:related_work}

\section{Battery-powered robotic platforms}
\label{lab:s1}


Hardware modularity is a way to make the robot adapt its resources to different environments and sensing operations.
By separating out power, computation, motor control and sensing a verity of capabilities can be tested \cite{RN13, RN2}.
Microrobots typically use infrared-based neighbor to neighbor distance sensing and communication \cite{RN3}.
While controlling a swarm or collective is mainly accomplished by means of active low power transceivers \cite{RN13, RN2}. 

Choosing the right locomotion resource can depend on different factors, moving in the most energy efficient way on a particular surface is often the determining factor.
On a flat surface, robots commonly use a two-wheeled differential drive design to not only move but allow for steering as well \cite{RN13, RN2}. 
In other designs overall cost is a decisive factor, Kilobot uses two vibrating motors for locomotion.
When the motors are activated the centripetal forces will generate a forward movement \cite{RN3}.
The GRITSBot does not use conventional DC motors, requiring encoders to estimate their speed. 
Instead by using stepper motors the speed can be set by changing the delay between steps. 
Estimating it's position therefore is reduced to simply counting steps \cite{RN2}. 

\section{Methods of energy harvesting for robotic platforms}
\label{lab:s2}

