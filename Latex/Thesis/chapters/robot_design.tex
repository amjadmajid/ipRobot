\chapter{Robot Design}

% - Size and weight/energy availability/speed trade-off model and 
% - Should the subsections more or less reflect the columns in the table?
% - No optical encoders because they require a lot of energy!

\section{Computation and Sensing}

The robot is designed around a modified WISP 5.0, utilizing the processor and backscatter communication.
It's based around a Texas Instruments MSP430FR5969 ultra low power microcontroller, operating at 16 MHz and featuring 64 KB FRAM, 2 KB SRAM and 40 IO. 

%The harvester IC is removed from the WISP in order to allow an external harvester to supply the power to the system by harvesting solar energy. 

The mainboard is the platform that connects everything together.
Most reference designs are extendable but this adds complexity, while weight and size are the main constraints of this robot design.
All the sensors can be interfaced trough a I2C connection.
To conserve as much power as possible each sensor can be enabled and disabled using dedicated pins.

For avoidance of obstacles, a Maxim Integrated MAX44000 proximity sensor was added of the robot facing forward.
This sensor can measure the amount of ambient light as well.
In addition the robot has a Bosch BMG250 Mems Gyroscope to measure yaw angle change and correct when necessary.

\section{Locomotion}

Two 6mm geared dc motors from Precision Microdrives are mounted in a 3d printed frame, directly under the mainboard of the IPR.
The motors are mounted diagonally opposite from each other making the robot as compact as possible.
A small wheel is mounted directly on the motor shaft and in the other diagonally opposite corners a free running wheel is mounted to the frame.
The speed of each motor can be controlled individually using a dual H-bridge, this differential drive configuration allows the robot to steer.

%The buck converter is can only supply 110mA of current.
On average the each motor consumes 38mA while running on a flat surface, which is well within the current limit that the buck converter can supply.
However when the motors are in not moving yet the inrush/start current is equal to the stall current, which is equal to 240mA for each motor.
This amount of current can not be supplied by the powersupply, using PWM to soft start the motor the average current can be limited and allowing a bulk capacitor to supply the voltage.

\section{Energy Harvesting}

% RF harvesting seems prommesing
% better to only use RF for communication and harvest energy from another source \cite{konstantioulos}
% wispcam long time to charge

%Photovoltaic cells used for indoor and outdoor energy harvesting, commonly have a different spectral sensitivity depending on the nature of their sources.
%Solar cells used in indoor applications need to have a high spectral sensitivity in the range of visible light (400 to 700nm).
%While for outdoor applications have a spectral sensitivity range can be broader, including near infrared  (500 nm to 1100 nm).
%Triple junction solar cells harvest in some cases up to 50\% of there energy out of the near infrared range.
% reference to paper tudelft!

% http://www.chip1stop.com/web/RUS/en/tutorialContents.do?page=009

The solar energy harvesting system is based around a Texas Instruments BQ25570. 
It includes a nano power boost charger with maximum power point tracking to extract the optimal amount of energy from the solar panel. 
% Why is this size capacitor chosen?
This harvested energy is stored in a 22mF - 4.5V supercapactor from AVX, chosen for it's low leakage current and small size.
The energy stored in supercap is a function of the capacitance and voltage difference between the plates, being equal to:

\begin{equation}
\label{eq:cap1}
E = \frac{1}{2}CV^{2}
\end{equation}

However, to be able to use the energy stored in the capacitor efficiently a voltage regular is required to supply a stable voltage to the connected loads.
The regulated output voltage is a lower threshold of the energy that can be used from the capacitor.
Lowering the output voltage allows for more energy to be used from the supercapacitor, but in general also lowers the power consumption of individual components. Rewriting Equation \ref{eq:cap1} results in Equation \ref{eq:cap2}:

\begin{equation}
\label{eq:cap2}
E = \frac{1}{2}C(V_{max} - V_{min})^{2}
\end{equation}

The Texas Insturments BQ25570 has a buck converter to efficiently regulate the capacitor voltage down to a system voltage of 2.2 Volt.

% The maximum speed that can be reached is dependent on the voltage supplied to the motors.
% The current consumed by the motors also decreases with a lower voltaged supplied. 
% The buck converter can supply up to 110mA of output current.
% Powering both motors and MCU should not exceed this current limit.

\section{Software}

% General library for reading and writing to i2c.

% How program consistency (ie progress) is maintained given the intermittend nature of the robot.

% A operating system for intermittend devices