%Small robotic platforms have become increasingly popular as an educational toy for kids and are widely used by researchers to study swarm behavior with a collective of robots.
Collectives of miniature robots are envisioned to have future applications in surveillance, search and rescue operations, and exploration. 
However, before these robots can become applicable in real-world applications, a fundamental challenge related to supply of energy needs to be addressed first.
That is, the operation time of small robots is currently limited by the energy storage in batteries.
Unfortunately, new advancements in batteries are not expected to happen anytime soon, the history shows that new battery technologies are slow emerging.
Therefore, this thesis proposes to replace the battery with an energy harvester and temporarily store harvested energy in a supercapacitor.
This results in a new phenomenon to be taken into consideration in designing a robot: frequent power failures due to the intermittent availability of energy.
Intermittency is currently not taken into account in the development process of a robot, and its effect on control techniques and accuracy of movement is, therefore, unexplored.
In this thesis a transiently-powered battery-free robot is developed that purely operates on harvested energy from light.
The robot is able to move with a 16\% power duty cycle, using a lighting setup consisting of four halogen lamps.
With the help of local feedback the robot is able to perform controlled movements, while variables stored in non-volatile memory enable the robot to save the movement progress across power cycles.
The movement accuracy of the transiently-powered robot is evaluated using tracking software to extract the exact path of movement from straight and circular motion recordings.	
The transiently-powered robot shows minimal increased deviation from its instructed path when compared to its battery powered equivalent, given an experimentally determined minimum on time of 0.3\,s.
The results prove the feasibility of a transiently-powered battery-free robot, clearing a potential path for self-sufficient and energy-autonomous small robots.