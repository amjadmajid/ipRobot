\chapter{Preliminaries}

In this chapter a simple model will capture the relation between power harvested and distance covered by the robot.
The design considerations for the robot will be explained based on minimal required capabilities. 
Followed by an evaluation of stepper motor based locomotion for a transiently powered robot.

\section{Transiently-powered Robot Model}
\label{sec:transient_model}

In this section a simple model will be derived showing the relation between size and weight of the robot, the amount of power that is harvested and stored and how much of this can be translated into linear movement.

\subsection{Modeling Assumptions}

Energy is harvested from an ambient source and stored in a supercapacitor.
To make efficient use of the energy stored in a supercapacitor a regular is required to supply a stable voltage to the connected loads.
In this case the loads are two identical dc motors which each drive wheel. \\
\\ \noindent
The following assumptions will be used to model the transiently powered robot:
\begin{itemize}
	\item The required power by the loads is greater than the incoming power, resulting in repeated power cycling of the robot.
	\item The amount of input power after conversion is constant due to the use of a controlled environment
	\item Since a regulator is used, the voltage in the capacitor will never fall below the operating voltage.
\end{itemize}

%The input power Pin, will be stored in a supercapacitor with capacitance C.


%The regulated output voltage is a lower threshold for the energy that can be used from the capacitor.
%The upper threshold is determined by the maximum voltage rating of the supercapacitor.
%Lowering the output voltage allows for more energy to be used from the supercapacitor, and also lowers the overall power consumption of individual components.
%The energy stored in supercapacitor is a function of the capacitance and the threshold voltage difference, being equal to:

\begin{equation}
\label{eq:cap2}
E = \frac{1}{2}C(V_{\max} - V_{\min})^{2}
\end{equation}






% 1 Incomming power V * I which scales with solar panel size

% 2 Maximum power point tracking (switchmode boost converter)

% 3 Stored in non-ideal supercapacitor with capcitiy C and a parrallel resistance Rleak and series resistance (ESR, typically small but not neglectable?)

% 123 determine chargetime

% Buck converter losses 

% Power consumed from source = Pcons = Ploss + Pweels

% Power P(t)  = F * v = T x omega

\subsection{Motor Dynamics}

\begin{circuitikz}
	%	\draw [help lines] (-1,-2) grid (12,5);
	
	% electrical equivalent circuit
	\draw (0,0) to[V, v_=$v$] (0,3);
	\draw (0,3) to[R, i>^=$i$, l=$R$] (3,3);
	\draw (3,3) to[L, l=$L$] (4,3);
	
	\draw (4,3) -- (5,3);
	\draw (5,0) to[V, v_=$U_i$] (5,3);
	\draw (0,0) -- (5,0);
	
	% drive
	\draw[fill=white] (4.85,0.85) rectangle (5.15,2.15);
	\draw[fill=white] (5,1.5) ellipse (.45 and .45);
	
	% shaft drive -> transmission
	\draw[fill=black] (5.45,1.45) rectangle (6.5,1.55);
	
	% momentum arrow of drive -> transmission
	\draw[line width=0.7pt,<-] (5.8,1) arc (-30:30:1);
	
	% moment of inertia
	\draw[fill=white] (7.5,1.59)
	ellipse (.15 and 0.4);
	\draw[fill=white, color=white] (6.9, 1.99)
	rectangle (8.49, 1.19);
	\draw (6.8,1.59) ellipse (.15 and 0.4);
	\draw (6.8,1.99) -- (7.5,1.99);
	\draw (6.8,1.19) -- (7.5,1.19);
	
	% shaft right from moment of inertia
	\draw[fill=black] (8.65,1.55) rectangle (8.9,1.65);
	
	% momentum arrow (left hand side of brake shoe)
	\draw[line width=0.7pt,->] (8.05,1.1) arc (-30:30:1);

	% descriptions inside graphic
	\draw (5.85,2.2) node {$\omega_A, M_A$};
	\draw (7.25,1.61) node {$J$};
	\draw (8.05,2.32) node {$M_R$};
	
\end{circuitikz}
\\
\noindent
The electrical dynamics of a dc motor can be described as:
\begin{equation}
	v = Ri + L \dot{i} + e
\end{equation}

\noindent
The mechanical dynamics of a motor can be described as:
\begin{equation}
\tau = J\dot{\omega} + B\omega + m
\end{equation}

\noindent
The electromechanical equations state that the back emf voltage is proportional to the angular velocity and the motor torque is proportional to the armature current:

\begin{equation}
    \begin{gathered}
		e = k_{e} \omega \\
		\tau = k_{t} i
    \end{gathered}
\end{equation}

\noindent
The electrical power consumed and mechanical power consumed will be equal to:

\begin{equation}
	\begin{gathered}
		p_{\text{e}} = vi \\
		p_{\text{m}} = \tau\omega
	\end{gathered}
\end{equation}
	

\subsection{Robot Dynamics}
The robot is modeled as a mass $m$, that is moved by two wheels with radius $r$, each connected directly to a motor.


The rolling friction between the wheels and the surface is equal to:
\begin{equation}
	F_{\text{k}} = \mu_{\text{k}}mg
\end{equation}

Therefore the torque applied to the motor due to rolling friction, as it is only present while the robot is moving relative to the surface and the equation becomes:

\begin{equation}
	T_{\text{ext}} = rF_{\text{k}} sgn(\omega)
\end{equation}

\noindent
The total mass is equal to the 

\section{Design Considerations}
\label{sec:design_considerations}

This section will shortly explain the main areas considered while designing the battery-less transiently-powered robot

% - Optimize or low power consumption (disable or standby sensors and motor ctrl when not used)
% - Minimal basic functionality (for simple swarm algorithms?) (but no power hungry components ie optical encoders or mouse sensors)
% - Low power communication
% - Navigation

% Extra extra:
% - Tradeoff chargetime and operation time

\begin{enumerate}
\item \textbf{Power} The robot should not rely on batteries, alternatively energy can be harvested from ambient sources and stored in a supercapacitor. 
Energy harvested in a controlled environment should charge the capacitor in under 10 seconds and stored energy should provide at least an operation time of 1 second, allowing the robot to make short controlled movements.

\item \textbf{Small form factor} By making the robot as small as possible, weight is kept to a minimum reducing the energy required for movement.
Secondly, the design of the robot should use low cost regular available parts, to make it convenient to build collectives of transiently-powered robots.

\item \textbf{Locomotion}
The energy used for movement will be the biggest part used from the total available energy budget.
To optimize the distance that can be covered with a single capacitor charge, an efficient locomotion type should be chosen for the movement on flat surfaces.

\item \textbf{Autonomous navigation}
During operation the robot will experience a very frequent loss of power. 
Despite regular power interruption the transiently-powered robot should be able to complete a movement with an acceptable error compared to the same robot being battery powered. 

\end{enumerate}


% RF harvesting seems prommesing
% Wispcam requires approx 4 seconds to harvest 20mJ at a distance of 20cm from the reader \cite{naderiparizi_rfid_2015}
% better to only use RF for communication and harvest energy from another source \cite{konstantioulos}

% Energy can be harvested from different sources

\section{Stepper motor-based Locomotion}

In order for a robot to move between locations without external feedback, accurate locomotion and basic odometry are required.
Wheel encoders are often used to determine the angular speed of each wheel, which can be used for to correct speed differences between the motors and can be integrated over time to acquire distance.
Miniaturizing encoders significantly reduces their resolution, and can be classified as power hungry when considering a small energy budget and active light source is used.
The GRITSBot~\cite{pickem_icra_2015} uses stepper motors to achieve accurate locomotion and basic odometry, as described in Section \ref{sec:locomotion}.
This section will further investigate the use of stepper motor based locomotion in combination with a transiently-powered robot.

\subsection{The Stepper Motor}
Stepper motors are a type of permanent magnet dc motor that start to rotate by supplying current to the motor coils in a specific direction.
The bipolar stepper motor used, requires current to be pulsed trough each of the four connections, in a fixed pattern, in order to rotate it forward or backward.
A Microcontroller (MCU) is used to keep track and instruct the next stepper motor position from a sequence of four.
The outputs of the MCU cannot supply enough current to drive a bipolar stepper motor, therefore a dual H-bridge is required to control the current trough each coil.

%TODO make new schematic stepper figure!
%http://homemaderobo.blogspot.nl/2012/03/stepper-motor.htm
\begin{figure}
	\centering
	\includegraphics[width=\textwidth]{pics/bipolar_stepper.png}
	\caption{Need better / simpler figure here!}
	\label{fig:bipolarstepper}
\end{figure}

\subsection{Control and Rotor Synchronization}

The only way to grantee that the teeth on the rotor will stay aligned with the coil is to keep the coil energized until the next position is instructed and the other coil is energized. 
On the first startup it can happen that the rotor is not aligned with the last position in the sequence.
When this happens there can be an error between one and three steps of instructing a new steps and the stepper moving to the next position.

In case the stepper motor is rotating and the power is removed, misalignment between the rotor and the last energized coil can occur.
While the rotor could be moving from one position to the next, it has not moved at all (undershoot) or it overshoots it's next position due to inertia of the rotating mass. 
To determine what would be more likely, undershooting or overshooting, an experiment has been preformed to determine the error in number of steps.

\subsection{Experimental setup}

Tiny permanent magnet bipolar stepper motors can be sourced from China, while they are frequently used in digital camera's~\cite{nidec_stepper_2017}.

A stepper motor is suspended and a needle glued to the motor shaft.
The needle rotates over a round piece of paper which is divided by markings in 20 steps, as can be seen from Figure \ref{fig:step_counting}.
First the rotor and coil are synchronized by moving four steps.
Then the position of the needle is recorded and the stepper motor is commanded to make one rotation equal to 20 steps.
After rotating 20 steps the power is removed from the coils and the needle position recorded when the needle is not rotating anymore.

\begin{figure}
	\centering
	\begin{subfigure}[b]{0.38\textwidth}
		\includegraphics[width=\textwidth]{pics/step_counting.jpg}
		\caption{Experimental setup for determining error in the number of counted steps}
		\label{fig:step_counting}
	\end{subfigure}	
	\quad
	\begin{subfigure}[b]{0.55\textwidth}
		\includegraphics[width=\textwidth]{pics/figure_intertia.png}
		\caption{}
		\label{fig:step_results}
	\end{subfigure}
	\caption{}
\end{figure}

\subsection{Result}

Figure \ref{fig:step_results} shows that stepper motor on average will overshoot, i.e. will do more steps than commanded when the power is removed.
This effect is probably caused by inertia of the rotor as it becomes more significant with increasing step frequency or speed of rotation.
However, while this experiment only shows the effect for an unloaded motor, it's likely that a synchronization error will occur when a transiently-powered robot would be powered using two stepper motors in differential drive.
After every power interrupt each motor first needs to be synchronized, which will result in the robot making a turn if the error between the motors is not equal.

%TODO values that calculate the current per motor?
Secondly, the current consumed by a stepper motor is constant and independent of the angular velocity of the motor.
The average current consumed is equal to: $\textrm{I} = V_{supply}/R_{coil}$.
Therefore running the motor at maximum speed, determined by is the most efficient, i.e translates the most electrical energy into kinetic energy.
However, the motor speed is inversely proportional to the motor's output torque and therefore the maximum speed is limited by the minimal required output torque.
%The current trough the coils is constant, so the faster the stepper motor changes step the more energy can be transformed into movement.
%Increasing the rotational speed of the stepper motor decreases the torque output of the motor.
%Therefore the speed is limited by the amount of torque required to preform the movement.





