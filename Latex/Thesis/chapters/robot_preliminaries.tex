\chapter{A Transiently-powered robot: Preliminaries}

In this chapter design requirements for the robot will be determined based on minimal required capabilities.



\section{Design Requirements}
\label{sec:design_requirements}

% RF harvesting seems prommesing
% Wispcam requires approx 4 seconds to harvest 20mJ at a distance of 20cm from the reader \cite{naderiparizi_rfid_2015}
% better to only use RF for communication and harvest energy from another source \cite{konstantioulos}

% Energy can be harvested from different sources


% - Small form factor
% - Weight of the robot
% - Power (should not rely on batteries)
% - Low voltage decreases power consumption of components and allows efficient use of the energy from the supercapacitor

% - Optimize or low power consumption (disable or standby sensors and motor ctrl when not used)
% - Minimal basic functionality (for simple swarm algorithms?) (but no power hungry components ie optical encoders or mouse sensors)
% - Low power communication
% - Navigation

Furthermore:
% - Single mainboard design
% - Off the chelf components
% - Swarm robots are typically limited to only operate on flat surfaces!
% - However should be able to be powered from batteries for testing and tuning!

% Extra extra:
% - Tradeoff chargetime and operation time


Current state of the art micro robotic platforms are not bigger than 4.4 cm, as can be seen from Table \label{tab:comparison_robot_platforms}.
Keeping the size of the robot to a minimum is beneficial, while the weight will scale accordingly and less energy will be required for movement.



Typically the amount of power that can be harvested 

The main requirement for the robot is that it should be battery-less, therefore it can only store a small amount of energy in a capacitor / small buffer.
The leakage current increases with the capacitance of the capacitor and the voltage stored in the capacitor.
As a result these require longer charge time

To make efficient use of the energy stored in a supercapacitor a regular is required to supply a stable voltage to the connected loads.
The regulated output voltage is a lower threshold for the energy that can be used from the capacitor.
The upper threshold is typically determined by the maximum voltage rating of the supercapacitor.
Lowering the output voltage allows for more energy to be used from the supercapacitor, but also lowers the overall power consumption of individual components.
The energy stored in supercapacitor is a function of the capacitance and the threshold voltage difference, being equal to:

\begin{equation}
\label{eq:cap2}
E = \frac{1}{2}C(V_{max} - V_{min})^{2}
\end{equation}

\section{Stepper motor based locomotion}

One of the major requirements for the robot was that it should have accurate locomotion and basic odometry.
Wheel encoders are typically used to estimate the angular speed of the wheels.
A combination of LEDs and photo diodes are used to measure the reflection of a rotating disk.
However, two LED's on constantly on could consume a lot of power.
The GRITSBot uses stepper motors to achieve accurate locomotion and basic odometry, as described in Section \ref{sec:locomotion}.
This section will further investigate possibility of using stepper motor based locomotion in combination with a transiently-powered robot.

\subsection{}
