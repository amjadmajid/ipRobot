% This document provides the style to be used for a MSc Thesis at the
% Parallel and Distributed Systems group
\documentclass[11pt,twoside,a4paper,openright]{report}

% math packages
\usepackage{amsmath}
\usepackage{amssymb}

% textblocks for title page
\usepackage[absolute]{textpos}

% use babel for proper hyphenation
\usepackage[british]{babel}

% Graphics: different for pdflatex or dvi output, choose one
%%\usepackage[dvips]{graphicx}
\usepackage[pdftex]{graphicx}
\usepackage{graphicx}

\usepackage{epstopdf}
\usepackage{rotating}
\usepackage{subcaption}

% FONT
\usepackage[scaled=.92]{helvet}
%\usepackage{times}
\usepackage{textcomp}
\usepackage{eurosym}

% for url's use "\url{http://www.google.com/}"
\usepackage{url}
\usepackage[plainpages=false]{hyperref} 

% Table
\usepackage{threeparttable, tablefootnote}
\usepackage{tabularx}
\usepackage{multirow}

% References
\usepackage{flushend}

% Tikz
\usepackage{tikz}
\usepackage{circuitikz}


% Information that will be filled in at various points in the report
\newcommand{\reportTitle}{Transiently-powered Battery-free Robot}
\newcommand{\reportAuthor}{Koen Schaper}
\newcommand{\reportEmail}{kpschaper@gmail.com}
\newcommand{\reportUrlEmail}{\href{mailto:\reportEmail}{\reportEmail}}
\newcommand{\reportMSC}{Embedded Systems} %{Embedded Systems}{Computer Engineering}{Computer Science}{Electrical Engineering}
\newcommand{\reportDate}{\today} %TODO: Dit is de datum van uitgifte van final versie aan de afstudeer commissie 
\newcommand{\presentationDate}{15 December 2017}
\newcommand{\graduationCommittee}{
prof. dr. K.G. Langendoen (chair) & Delft University of Technology \\
dr. Przemys\l{}aw Pawe\l{}czak (supervisor) & Delft University of Technology \\
dr. Javier Alonso-Mora & Delft University of Technology \\
} % The order of listing the names: Graduation prof, supervisor(s), others ordered by title + alphabetical 
%examples: 
%prof. dr. ir. H. J. Sips (chair) & Delft University of Technology \\ 
%ir. dr. D. H. J. Epema           & Delft University of Technology \\ 
\newcommand{\reportAbstract}{
	%Small robotic platforms have become increasingly popular as an educational toy for kids and are widely used by researchers to study swarm behavior with a collective of robots.
	Collectives of miniature robots are envisioned to have future applications in surveillance, search and rescue, and exploration. 
	Before these robots can become applicable in real world applications, a fundamental challenge related to the supply of energy needs to be addressed.
	The operation time of small robots is currently limited by the energy stored in batteries.
	New advancements in batteries are not expected to happen anytime soon, as history shows that new battery technologies are slow emerging.
	This thesis proposes to replace the battery with an energy harvester and temporarily store harvested energy in a supercapacitor, which creates a new phenomenon to be taken into consideration; the intermittent availablility of energy results in frequent power interrupts.
	Intermittentcy is currently not taken into account when developing a robot, and its effect on control techniques and locomotion are unexplored.
	In this work a transiently-powered battery-free robot is developed that purely operates from harvested energy.
	Energy is harvested from light using a lighting setup consisting of four halogen lamps, making the robot is able to move with a 16\% power duty cycle.
	With the help of local feedback the robot is able to preform controlled movements, while peristent state variables enable the robot to save the movement progress across power cycles.
	The movement accuracy of the transiently-powered robot is evaluated by recording straight and curved movements, and tracking software is used to extract the exact path of movement.
	The transiently-powered robot shows minimal increased deviation from its instructed path when compared to its battery powered equal, given an experimentially determined minimum interrupt period of 0.3\,s.
	The results prove the feasablility of a transiently-powered battery-free robot, while it makes small robots self sufficient and energy autonomous.
}
\newcommand{\reportKeywords}{TODO KEYWORDS}

% For pdflatex
\pdfinfo{
   /Author (\reportAuthor)
   /Title  (\reportTitle)
   /Keywords (\reportKeywords)
}

\begin{document}

\pagenumbering{alph}
\pagestyle{empty}


% FRONTCOVER
\include{template/frontcover}

%%%%%%%%%%%%%%%%%%%%%%%%%%%%%%%%%%%%%%%%%%%%%%%%%%%%%%%%%%%%%%%%%%%%%%%%%%%%%%%
\hoffset=1.63cm
\oddsidemargin=0in
\evensidemargin=0in
\textwidth=5in

%%%%%%%%%%%%%%%%%%%%%%%%%%%%%%%%%%%%%%%%%%%%%%%%%%%%%%%%%%%%%%%%%%%%%%%%%%%%%%%
\parindent=1em

% EMPTY PAGE
\cleardoublepage

\pagestyle{plain}
\pagenumbering{roman}
\setcounter{page}{1}

% TITLE PAGE: page i (hidden)
\include{template/titlepage}

% GRADUATION DATA AND ABSTRACT: pages ii and iii (hidden)
\include{template/graduationdata}
%\setcounter{page}{4}

% EMPTY PAGE: page iv
\cleardoublepage

% OPTIONAL QUOTATION: page v
%\include{quotation}
% EMPTY PAGE: page vi
%\cleardoublepage

% PREFACE: page v
\chapter*{Preface}
\addcontentsline{toc}{chapter}{Preface}
TODO MOTIVATION FOR RESEARCH TOPIC

\vspace{1\baselineskip}

\noindent
TODO ACKNOWLEDGEMENTS

Thanks to the Nuon Solar Team for letting me borrow one of their test panels for their 2017 Nuna9 solar car.

\vspace{1\baselineskip}

\noindent
Koen Schaper

\vspace{1\baselineskip}

\noindent
Delft, The Netherlands

\noindent
\today

% EMPTY PAGE: page vi
\cleardoublepage

% TABLE OF CONTENTS: starting at page vii
\tableofcontents

\cleardoublepage

\pagenumbering{arabic}
\setcounter{page}{1}

% INTRODUCTION: page 1
\chapter{Introduction}
\label{chp:introduction}

\section{Motivation}

what kind of small robots are there
- Nanobots
- self asembling bots
- robot platforms for swarm





There is an increasing research interest in small autonomous robots, since they have the potential to 



Gathering information in case of disaster
Mobile sensing for hard to reach or inspect places
Collaboration increase fault tollerance


Robots need to explore their surroundings, sense, process and react accordingly.	


Small sizes allow robots to work in areas that would otherwise be unreachable. 
However these applications will require new sensing, control and navigation strategies to make use of the increasingly resource constrained robots.

Small robotic platforms still rely on batteries as a source of power, since electric motors consume considerably more energy compared to computation and sensing.
The currently most commonly used lithium-ion batteries, have a high energy density but still limit the operation time of the robot.
When the voltage of the battery drops below a certain level, indicating that the battery is almost empty, the robot needs to move to a charging station before it runs out of energy.
Another option could be to replace the battery, but this would still require the robot to return to a station.
Replacing the battery with an energy harvesting system would make the robot energy-autonomous. 

% what are the challenges???

\section{Problem statement}

% what is the goal??
Let's say you want to have it as a backup system.
Or it's a new approach to current robotic platforms.

Challenge:
Current software used for robots with a stable power supply cannot be ported directly to this new platform.
So what changes are required?

Intermittently powered robots, that purely rely on harvested power are currently non existent.
The intermittent availability of energy creates new challenges in control, navigation and actuation. 
This research will explore the feasibility of an autonomous battery-less robot, focusing on the development of such a robot and especially porting a complex task like navigation to it. 
Therefore the main question this work will try to answer is: \\
\textit{How to make a transiently powered robot navigate autonomously?}

\section{System Description}



\section{Contributions}

New approach to supplying energy to the robot which adds new challenges in computation

* first study of intermittently powered robots, unaware of any of such platforms besides hypothetical nanobots that supposed to be harvesting energy from movement or organic stuff.

* Design of a small robot with accurate locomotion, basic sensing and communication.

* Implementation of a complex task like navigation to this intermittently powered robot

* Evaluation of the performance of this new approach compared to a battery powered robot.

\section{Outline}
What will be discussed in the coming chapters


\vspace{1\baselineskip}

\noindent
TODO ORGANISATIONAL DESCRIPTION OF THESIS



% CHAPTERS ... For instance: History/Prior Work, Design/Implementation, Experiments
\chapter{Related Work}
\label{chp:related_work}

This chapter will provide background information about current state of the art transiently powered systems. The advantages and disadvantages of different electrical storage types are compared. A short summary of current miniature robotics platforms is given and commonly used locomotion types. Finally different methods that try to ensure continuous operation will be discussed.

\section{Transiently-powered Systems}
\label{sec:tp_systems}

% - Roughly everything that is powered fron a Energy harvester

Example of energy harvesting: Prolonged energy harvesting for ingestible devices~\cite{plonski_tranro_2016}
Drug delivery
\\
Converting a Plant to a Battery and Wireless Sensor with Scatter Radio and Ultra-Low Cost

%Other sources available for exploration are often limited by the application. Secondly, most sources can be scarce or completely absent during prolonged time intervals of the day as well \cite{RN15}. 

Fully programmable RFID platforms have been developed to exploring the combination of sensing, computation and communication, while allowing battery-less operation by harvesting RF energy~\cite{sample_transim_2008}.
The amount of energy collected from RF signals is very small and decreases with the distance of the device to the transmitter.
The harvested energy is typically stored in a capacitor, where larger capacitors can buffer more energy and smaller capacitors have the advantage of shorter charge times~\cite{gummerson_mobisys_2010}.
For longer, complex operations the energy budged needs to be evaluated carefully.
To store the energy an appropriate size storage capacitor needs to be selected~\cite{naderiparizi_rfid_2015}.

% - Short intro into persistent framwork: checkpointing etc
% mementos
% chain
% ratchet

\section{Energy Supply}
\label{sec:energy_supply}

% Explain battery vs supercapacitor

Comparing li-ion batteries with super-capacitors there are some big differences.
Supercapacitors do not need any special charging scheme and circuity for charging, except for overcharging protection.
Secondly, super-capacitors do not require any particular current profile, the energy can be stored at any rate and when the energy is required it can be extracted at any power level.
Operating a li-ion battery outside of it's recommended operating conditions can severely reduce a batteries lifetime and result in overheating or even explosion of the battery.
Batteries will seldom withstand more than one thousand complete charge/discharge cycles.
Super-capacitors used under extreme condition's, are not likely to explode but instead rupture.
While the biggest disadvantages of super-capacitors is their low energy density and high price, their lifetime is typically hundred thousands of charge/discharge cycles.

Li-Ion Battery-Supercapacitor Hybrid Storage System for a Long Lifetime, Photovoltaic-Based Wireless Sensor Network	\cite{ongaro_pwre_2012}
Reincarnation in the Ambiance: Devices and Networks with Energy Harvesting \cite{prasad_comst_2014}



\section{Small Robotic Platforms}
\label{sec:robotic_platforms}

%TODO tell that these robots are used for studing swarm behavior (inspired by nature)
% require communication to apply different algorithms
% need to be low cost and size are key factors for allowing scalability
% 
%Main characteristics : https://link.springer.com/article/10.1007/s11721-012-0075-2
%robots are autonomous;
%robots are situated in the environment and can act to modify it;
%robots’ sensing and communication capabilities are local;
%robots do not have access to centralized control and/or to global knowledge;
%robots cooperate to tackle a given task.

Low cost robotic platforms have been developed to tackle a variety of challenges anonymously.
Miniature robots can be used for inspection in difficult to reach places, operating like mobile sensing units.
Hardware modularity is a way to make the robot adapt its resources to different environments and sensing operations.
By separating out power, computation, motor control and sensing a verity of capabilities can be tested~\cite{sabelhaus_icra_2013, pickem_icra_2015, kim_iros_2016}.
Microrobots typically use infrared-based neighbor to neighbor distance sensing and communication~\cite{rubenstein_icra_2012, pickem_icra_2015, kim_iros_2016}.
While controlling a swarm or collective is mainly accomplished by means of active low power transceivers~\cite{sabelhaus_icra_2013, pickem_icra_2015, kim_iros_2016}. 

%TODO reference table

\section{Locomotion}
\label{sec:locomotion}
%TODO tell somthing about their accuracy!

Choosing the right locomotion resource can depend on different factors, moving in the most energy efficient way on a particular surface is often the determining factor.
On a flat surface, robots commonly use a two-wheeled differential drive design to not only move but allow for steering as well~\cite{sabelhaus_icra_2013, pickem_icra_2015}.
The motor shafts of the motors on the mROBerTO directly contact the surface, eliminating the need for wheels and simplifying the design.
A tiny 1/8" ball is used as a third support point in the front of the robot~\cite{kim_iros_2016}.
The GRITSBot does not use conventional DC motors, requiring encoders to estimate their speed. 
Instead by using stepper motors the speed can be set by changing the delay between steps. 
Estimating it's position therefore is reduced to simply counting steps~\cite{pickem_icra_2015}.  
Overall cost can be a decisive factor, therefore the Kilobot uses two vibrating motors for locomotion combined with three thin legs.
When the motors are activated the centripetal forces will generate a forward movement, which can be explained using the slip-stick principle~\cite{rubenstein_icra_2012}.
Other locomotion types are biologically inspired, the HARM-VP is small scale piezoelectric driven quadrupled robot~\cite{baisch_iros_2013}.
Each leg as two degrees of freedom, it can move up and down, as well as forward and backward.

\section{Continuous Operation}
\label{sec:continous_operation}
%Battery replenishment

%TODO include battery tosti iron anecdote
Typically the operation time is extended by regularly checking the remaining energy in the battery and move to a recharging station before the robot runs out of energy~\cite{pickem_icra_2015, rubenstein_icra_2012}.
As an alternative to quickly recharging, the battery can also be swapped automatically when the robot moves into the docking station~\cite{kemal_mech_2015}.
Another work shows a robot which is able to swap it's primary battery using a six degree-of-freedom manipulator, used to grab the dead battery and plug it into a wireless recharging charging station~\cite{zhang_conel_2013}.
Using direct wireless power to replace or supplement to a batteries energy is shown in~\cite{karpelson_icra_2014}, however the robot can only operate or recharge while remaining in close proximity to a transmitter. 
In these cases the robots are highly reliant on an infrastructure to allow for continuous autonomous operation.
This can be a severe constraint if the robot moves out of reach or needs to operate in a area where this infrastructure is not present. Persistent operation can also be achieved by harvesting renewable energy, particularly solar energy to complement to the robots internal energy source. 
To remove weight from the robot, in~\cite{bruhwiler_iros_2015} the solar energy is used directly without any type of energy buffer. 
A drawback of this method is that the incoming solar energy should greater or equal to the energy required for operation. 
This approach has only been tested for basic locomotion and did not yet combine any form of sensing or control.

% - Provide overview table robots smaller than 15*15cm

% For each of these cases you need to provide numbers: 
% level of autonomy (does the robot does all by itself or relies on external processing)
% does autonomy fall under 
% charging time

% Add missing "new" robots

\begin{table}[t]
	\centering
	\resizebox{\columnwidth}{!}{%
		\begin{threeparttable}
			\caption{Comparison of small robotic platforms}
			\label{tab:comparison_robot_platforms}
	 		\begin{tabular}{|l l l l l l l|} 
				\hline 
	 			Robot & Locomotion & Size & Weight & Energy Storage & Recharge \\ 
	 			(Cost) &           &      &        & Size \& Life   & Method   \\ 
	 			\hline\hline
	 			IPR  & wheel, & 4.0 cm & 21 g & 0.006 mAh, 1 s & solar \\
	 			(\euro59) & 25 cm/s &  &      &                &       \\
	 			HAMR-VP\textsuperscript{1} \cite{bruhwiler_iros_2015}& legged, & 4.4 cm & 2.3 g & 8 mAh, 3 m & manual \\
	 			(N/A) & 1 cm/s                                                 &        &       &            &        \\
	 			Roverables \cite{dementyev_uist_2016} & wheel, & 4.0 cm & ?? & 100 mAh, 45 m & inductive \\
	 			(N/A) & ??                                     &        &    &               &           \\                             
	 			Zooids \cite{legoc_uist_2016}& wheel, & 2.6 cm & 12 g & 100 mAh 1 h & manual \\
	 			(\euro43) & 50 cm/s &                 &        &      &             &        \\                    
	 			mROBerTO \cite{kim_iros_2016} & motor shaft, & 1.5 cm & ?? & 120 mAh, 1.5 h & manual \\
	 			(\euro52 \textsuperscript{2}) & 15 cm/s      &        &    &               &         \\
	 			GRITSBot \cite{pickem_icra_2015} & wheel, & 3 cm & ?? & 150 mAh, 1 h & contact \\
	 			(\euro43\textsuperscript{2}) & 25 cm/s    &      &    &              &         \\ 
	 			Kilobot \cite{rubenstein_icra_2012} & vibration, & 3.3 cm & 17.6 g & 160 mAh, 3 h & manual \\
	 			(\euro43\textsuperscript{2}) & 1 cm/s            &        &    &              & bulk \\
	 			TinyTerp \cite{sabelhaus_icra_2013} & wheel, & 1.8 cm & ?? & 50 mAh, 1 h & manual \\
	 			(\euro43) & 50 cm/s                 &        &        &    &             &        \\
				\hline
			\end{tabular}
			\begin{tablenotes}
				\small
				\item [1] Modified to include on-board power, sensing and control.
				\item [2] Cost of parts
			\end{tablenotes}
		\end{threeparttable}
	}
\end{table}

\chapter{Preliminaries}

In this chapter a simple model will capture the relation between power harvested and distance covered by the robot.
The design considerations for the robot will be explained based on minimal required capabilities. 
Followed by an evaluation of stepper motor based locomotion for a transiently powered robot.

\section{Energy Source Selection}
% RF harvesting seems prommesing
% Wispcam requires approx 4 seconds to harvest 20mJ at a distance of 20cm from the reader \cite{naderiparizi_rfid_2015}
% better to only use RF for communication and harvest energy from another source \cite{konstantioulos}

In this section the charge time of two different sources of ambient energy will be compared, RF and Solar.
To determine which source is the most suitable, their performance will be evaluated for different sources and distances from the source.
The average input power $P_{\text{in}}$ will be determined from the time $t_{\text{charge}}$, to charge a capacitor.

\begin{equation}
	P_{\text{in}} = \frac{E_{\text{cap}}}{t_{\text{charge}}}
\end{equation}

\noindent
Given a capacitor with capacity $C$, a minimum voltage level $V_{\min}$ and a maximum voltage level $V_{\max}$, the energy stored in a capacitor can be determined using:

\begin{equation}
	E_{\text{cap}} = \frac{1}{2}C(V_{\max} - V_{\min})^{2}
\end{equation}

\subsection{Energy Harvesting and Storage}
\label{sec:dai_energy_harvesting}
Energy is harvested using a Texas Instruments BQ25570 energy harvester~\cite{bq25570_2017}, which includes a nanopower boost charger with maximum power point tracking to extract the optimal amount of energy. 
The harvested energy is stored in a 22\,mF - 4.5\,V supercapacitor from AVX~\cite{avx_bestcap_2017}, chosen for its low leakage current and small size.
The Texas Instruments BQ25570 has a buck converter to efficiently regulate the capacitors voltage down to the system voltage of 2.2\,V.
External resistors are used to program voltage thresholds, allowing to automatically enable and disable the buck converter based on minimum and maximum thresholds.
The minimum threshold is set to 2.2\,V and the maximum threshold is set to 4.2\,V.
Therefore the energy stored in the capacitor before the buck converter is enabled is equal to:

\begin{equation}
\label{eq:cap2}
E = \frac{1}{2} 0.022 (4.4 - 2.2)^2 = 53.24 mJ
\end{equation}

%Additionally the resistors are used to set the overvoltage protection and the buck converter output voltage.
%The minimal supply voltage is determined by the component with the highest minimal voltage requirement, in this case 2.0\,V.
%To make sure that a small drop in system voltage would not create instability a margin of 0.2\,V was added, resulting of a system voltage of 2.2\,V.

\subsection{Determining the Charge Time}

The buck converter of the energy harvester is automatically enabled when the maximum voltage threshold is reached.
A load is connected to the output of the buck converter to quickly drain the energy from the capacitor.
In this case load is chosen such that the power consumed by the load $P_{\text{load}} >> P_{\text{in}}$.
By connecting a Saleae logic analyzer~\cite{saleae_2017} to the output of the buck converter, the period of the power cycles can be recorded.
The time that the output of the buck converter is disabled will be equal to the time to charge the capacitor from the minimum to the maximum threshold.


\subsection{Energy Harvesting from RF}

\subsubsection{Modified WISP}
To be able to connect an external harvester, a WISP 5~\cite{sample_transim_2008} was modified.
The integrated energy harvester, the storage capacitor and the diode to bypass the harvester, were removed from a WISP.
A wire was soldered to the input pin pad of the now removed harvester on the WISP PCB.
This wire was connected directly to the input of the external energy harvester.

\subsection{Measurements}
Energy was provided to the WISP using a Impinj Speedway R1000 RFID reader~\cite{impinj_eol_2017, indy_r1000_2017}.
This reader was connected to a Laird S90028PCR antenna~\cite{laird_s9028pcr_2017}.
The WISP is positioned 25, 35 and 45 cm away from the reader and the charge time was recorded.

\begin{table}[t]
	\centering
	\caption{The average charge time with RF}
	\label{tab:res_rf_harvest}
	\begin{tabular}{|l||l|l|l|}
		\hline
		Distance & 25\,cm & 35\,cm & 45\,cm \\
		\hline \hline
		Average charge time & 49.1\,s & 61.1\,s & 164.8\,s \\
		Average input power & 1.086\,mW & 0.871\,mW & 0.323\,mW \\
		\hline
	\end{tabular}
\end{table}

\subsubsection{Results}
The time to charge the capacitor is more than 49s, see Table \ref{tab:res_rf_harvest}.
As the WISP is placed further away from the reader the charge times increase significantly.
While the distance increases more power is directed away from the WISP due to reflections of the signal.

\subsection{Energy Harvesting from Light}

In this section the charge time of small solar panels under different light sources is evaluated.
There is not always enough sunlight available to charge the robots in a acceptable time.
A lighting setup needs to be created that provides a reasonable amount of uniform light to the area where the robot moves around.
To accurately measure the power that is harvested from each solar panel all the experiments were preformed in a darkroom at TU Delft Embedded Software Lab.

\subsubsection{Solar panels}
Three different solar panels were tested in this experiment, each different in material, efficiency and panel size, as can be seen from Table \ref{tab:solar_panels}.

\begin{table}[t]
	\centering
	\resizebox{\columnwidth}{!}{%
		\begin{threeparttable}
			\caption{Specification of the solar panels tested in the experiment.}
			\label{tab:solar_panels}
			\begin{tabular}{|l|l|l|l|}
				\hline
				& Material & Efficiency (\%) & Dimensions (mm) \\
				\hline \hline
				Banggood~\cite{bangood_solar_2017}& Poly-Si & 17 & 40x30 \\
				INYS SLMD121H04L-ND~\cite{ixolar_slmd121h04l_2017}\textsuperscript{1}& Mono-Si & 22 & 43x34 \\
				Azurspace 3G28C~\cite{azurspace_3g28c_2017}& Triple Junction GaAs& 28 & 80x40 \\
				\hline
			\end{tabular}
			\begin{tablenotes}
				\small
				\item [1] Two panels in parallel
			\end{tablenotes}
		\end{threeparttable}
	}
\end{table}

%TODO Why these lamps?
\subsubsection{Lamps}
Low cost solar simulators can consist of a combination of LED and halogen light bulbs to simulate sunlight and are used to test the performance of solar panels~\cite{grandi_tia_2014}.
However, in this case the goal is to have a controlled uniform lighting environment where the robots have roughly constant charge times.
Solar panels do not only harvest energy from the visual light spectrum but harvest almost at least as much from the infrared light spectrum, therefore not only light but also heat will shorten the charge time.
Halogen lamps have a lower color temperature than the sun but also emit waves far into the infrared spectrum.
The light sources used in this experiment are a 60\,W halogen bulb, a 120\,W halogen halogen bulb and two 150\,W IR heat lamps where one is colored red.

\subsubsection{Measurements}
Three charge time measurements were preformed, each lamp was positioned 10\,cm, 30\,cm and 50\,cm from the solar panels.
To have a reference the charge times were also measured on a sunny afternoon. 
Additionally, for these three distance the temperature was measured at the solar panel using a K-type thermocouple supplied with an Extech EX330 multimeter and the light intensity using the luxmeter on a MASTECH MS8229 multimeter.

\subsubsection{Results}
% No difference between the heatlamps in power consumed
% Halogen distributes the light more even
% Panel from nuna
% Refer to appendix for temperature and light data?

Both the temperature and illumination increase by decreasing the distance between the light source and the solar panel. 
Secondly, increasing the output power of the lamp increases temperature and illumination as well. 
However the charge times 
A thing to note is that the both the 60\,W and 150\,W IR lamps have a spherical design. This creates a uneven circular shadowing pattern on the surface the lamps are shining on, which becomes more significant on the bigger distances in this experiment.
The 120\,W halogen lamp has a tubular design and in combination with the light fixture most of the light is reflected down with minimal shadowing of the lamp resulting in a more even light distribution.

\begin{figure}
	\centering
	\begin{subfigure}[b]{0.62\textwidth}
		\includegraphics[width=\textwidth]{pics/light_experiment_temp.png}
		\caption{Temperature at different distances}
		\label{fig:light_temp}
	\end{subfigure}
	\begin{subfigure}[b]{0.62\textwidth}
		\includegraphics[width=\textwidth]{pics/light_experiment_lux.png}
		\caption{Light intensity at different distances}
		\label{fig:light_lux}
	\end{subfigure}
	\caption{}
\end{figure}


\begin{figure}
	\centering
	\begin{subfigure}[b]{0.62\textwidth}
		\includegraphics[width=\textwidth]{pics/light_experiment_figure1.png}
		\caption{Ebay panel}
		\label{fig:light_exp1}
	\end{subfigure}
	\begin{subfigure}[b]{0.62\textwidth}
		\includegraphics[width=\textwidth]{pics/light_experiment_figure2.png}
		\caption{IXYS SLMD121H04L-ND}
		\label{fig:light_exp2}
	\end{subfigure}
	\begin{subfigure}[b]{0.62\textwidth}
		\includegraphics[width=\textwidth]{pics/light_experiment_figure3.png}
		\caption{Azurspace 3G28C}
		\label{fig:light_exp3}
	\end{subfigure}
	\caption{The performance of three different solar panels for different distances from different light sources. The data charge times for the last two are normalized with respect to the surface covered by the first panel.}
\end{figure}

\section{Locomotion Selection}

In order for a robot to move between locations without external feedback, accurate locomotion and basic odometry are required.
%TODO DEFINE AND REFERENCE
Wheel encoders are often used to determine the angular speed of each wheel, which can be used to correct speed differences between the motors and can be integrated over time to acquire distance.
Miniaturizing encoders significantly reduces their resolution, and can be classified as power hungry when considering a small energy budget and active light source is used.
%TODO NEED REFERENCE FOR THIS CLAIM

\subsection{Stepper motor-based Locomotion}

The GRITSBot~\cite{pickem_icra_2015} uses stepper motors to achieve accurate locomotion and basic odometry, as described in Section \ref{sec:locomotion}.
This section will further investigate the use of stepper motor based locomotion in combination with a transiently-powered robot.

\subsubsection{Operation of a Stepper Motor}
Stepper motors are permanent magnet dc motors that start to rotate by supplying current to the motor coils in a specific direction.
The bipolar stepper motor used, requires current to be pulsed trough each of the four connections, in a fixed pattern, in order to rotate it forward or backward.
A Microcontroller (MCU) is used to keep track and instruct the next stepper motor position from a sequence of four.
The outputs of the MCU cannot supply enough current to drive a bipolar stepper motor, therefore a dual H-bridge is required to control the current trough each coil.

%TODO make new schematic stepper figure!
%http://homemaderobo.blogspot.nl/2012/03/stepper-motor.htm
\begin{figure}
	\centering
	\includegraphics[width=\textwidth]{pics/bipolar_stepper.png}
	\caption{Need better / simpler figure here!}
	\label{fig:bipolarstepper}
\end{figure}

\subsubsection{Power Consumption}
%TODO values that calculate the current per motor?
%TODO ADD CURRENT MEASUREMENTS @ 2.2 v
The current consumed by a stepper motor is constant and independent of the angular velocity of the motor.
The average current consumed is equal to: $\textrm{I} = V_{\text{supply}}/R_{\text{coil}}$.
Therefore running the motor at maximum speed, translates the most electrical energy into kinetic energy.
However, the motor speed is inversely proportional to the motor's output torque and therefore the maximum speed is limited by the minimal required output torque.
%The current trough the coils is constant, so the faster the stepper motor changes step the more energy can be transformed into movement.
%Increasing the rotational speed of the stepper motor decreases the torque output of the motor.
%Therefore the speed is limited by the amount of torque required to preform the movement.

\subsubsection{Control and Rotor Synchronization}

The only way to grantee that the teeth on the rotor will stay aligned with the coil, is to keep the coil energized until the next position is instructed and succeeding coil is energized. 
On the first startup the rotor may not be aligned with the last position in the sequence of four.
As a result an error between one and three steps can occur before the energized coil and rotor are synchronized.

In case the stepper motor is rotating and the power is removed, misalignment between the rotor and the last energized coil can occur.
While the rotor could be moving from one position to the next, it has not moved at all (undershoot) or can continue to move to the next position due to inertia of the rotating mass (overshoot). 
To determine what would be more likely, undershooting or overshooting, the following experiment has been preformed to determine the error in the number of steps.

\subsubsection{Experimental setup}

Tiny permanent magnet bipolar stepper motors can be sourced from Ebay.com, while they are frequently used in digital camera's~\cite{nidec_stepper_2017}.
A stepper motor is suspended and a needle glued to the motor shaft.
The needle rotates over a round piece of paper which is divided by markings in 20 steps, as seen from Figure \ref{fig:step_counting}.
First the rotor and coil are synchronized by moving four steps, and the position of the needle is visually recorded and written down.  
Then the stepper motor is commanded to make one rotation equal to 20 steps.
After rotating 20 steps the power is removed from the coils and the needle position recorded when the needle is not rotating anymore.

\begin{figure}
	\centering
	\begin{subfigure}[b]{0.38\textwidth}
		\includegraphics[width=\textwidth]{pics/step_counting.jpg}
		\caption{Experimental setup for determining error in the number of counted steps}
		\label{fig:step_counting}
	\end{subfigure}	
	\quad
	\begin{subfigure}[b]{0.55\textwidth}
		\includegraphics[width=\textwidth]{pics/figure_intertia.png}
		\caption{}
		\label{fig:step_results}
	\end{subfigure}
	\caption{}
\end{figure}

\subsubsection{Stepper Motor Inertia Result}

Figure \ref{fig:step_results} shows the result of the experiment.
The stepper motor on average will overshoot, i.e. will do more steps than commanded when the power is removed.
This effect is becomes more significant with increasing step frequency or speed of rotation.
However, while this experiment only shows the effect for an unloaded motor, it is likely that a synchronization error will occur when a transiently-powered robot would be powered using two stepper motors in differential drive.
After every power interrupt each motor first needs to be synchronized, which will result in the robot making a turn if the error between the motors is not equal.

%TODO Write conclusion here! + link to rest of thesis

\subsection{DC Motor locomotion}

%TODO expain the general operation of a dc motor!
%TODO tell something about the motors considered
%TODO Show measurement / current profile of a dc motor


%TODO tell something about the maximum speed that can be achieved

On average the each motor consumes 38\,mA while running on a flat surface, which is well within the current limit that the buck converter can supply.
However when the motors are in not moving yet the start current is approximately equal to the stall current, which is equal to 240\,mA.
This amount of current can not be supplied by a most regulators, PWM can be used to reduce the average current allowing a bulk capacitor to supply the voltage.


\section{Transiently-powered Robot Model}
\label{sec:transient_model}

In this section a simple model will be derived showing the relation between size and weight of the robot, the amount of power that is harvested and stored and how much of this can be translated into linear movement.

\subsection{Modeling Assumptions}

Energy is harvested from an ambient source and stored in a supercapacitor.
To make efficient use of the energy stored in a supercapacitor a regular is required to supply a stable voltage to the connected loads.
In this case the loads are two identical dc motors which each drive wheel. \\
\\ \noindent
The following assumptions will be used to model the transiently powered robot:
\begin{itemize}
	\item The required power by the loads is greater than the incoming power, resulting in repeated power cycling of the robot.
	\item The amount of input power after conversion is constant due to the use of a controlled environment
	\item Since a regulator is used, the voltage in the capacitor will never fall below the operating voltage.
\end{itemize}

%The input power Pin, will be stored in a supercapacitor with capacitance C.


%The regulated output voltage is a lower threshold for the energy that can be used from the capacitor.
%The upper threshold is determined by the maximum voltage rating of the supercapacitor.
%Lowering the output voltage allows for more energy to be used from the supercapacitor, and also lowers the overall power consumption of individual components.
%The energy stored in supercapacitor is a function of the capacitance and the threshold voltage difference, being equal to:








% 1 Incomming power V * I which scales with solar panel size

% 2 Maximum power point tracking (switchmode boost converter)

% 3 Stored in non-ideal supercapacitor with capcitiy C and a parrallel resistance Rleak and series resistance (ESR, typically small but not neglectable?)

% 123 determine chargetime

% Buck converter losses 

% Power consumed from source = Pcons = Ploss + Pweels

% Power P(t)  = F * v = T x omega

\subsection{Motor Dynamics}

\begin{circuitikz}
	%	\draw [help lines] (-1,-2) grid (12,5);
	
	% electrical equivalent circuit
	\draw (0,0) to[V, v_=$v$] (0,3);
	\draw (0,3) to[R, i>^=$i$, l=$R$] (3,3);
	\draw (3,3) to[L, l=$L$] (4,3);
	
	\draw (4,3) -- (5,3);
	\draw (5,0) to[V, v_=$U_i$] (5,3);
	\draw (0,0) -- (5,0);
	
	% drive
	\draw[fill=white] (4.85,0.85) rectangle (5.15,2.15);
	\draw[fill=white] (5,1.5) ellipse (.45 and .45);
	
	% shaft drive -> transmission
	\draw[fill=black] (5.45,1.45) rectangle (6.5,1.55);
	
	% momentum arrow of drive -> transmission
	\draw[line width=0.7pt,<-] (5.8,1) arc (-30:30:1);
	
	% moment of inertia
	\draw[fill=white] (7.5,1.59)
	ellipse (.15 and 0.4);
	\draw[fill=white, color=white] (6.9, 1.99)
	rectangle (8.49, 1.19);
	\draw (6.8,1.59) ellipse (.15 and 0.4);
	\draw (6.8,1.99) -- (7.5,1.99);
	\draw (6.8,1.19) -- (7.5,1.19);
	
	% shaft right from moment of inertia
	\draw[fill=black] (8.65,1.55) rectangle (8.9,1.65);
	
	% momentum arrow (left hand side of brake shoe)
	\draw[line width=0.7pt,->] (8.05,1.1) arc (-30:30:1);
	
	% descriptions inside graphic
	\draw (5.85,2.2) node {$\omega_A, M_A$};
	\draw (7.25,1.61) node {$J$};
	\draw (8.05,2.32) node {$M_R$};
	
\end{circuitikz}
\\
\noindent
The electrical dynamics of a dc motor can be described as:
\begin{equation}
v = Ri + L \dot{i} + e
\end{equation}

\noindent
The mechanical dynamics of a motor can be described as:
\begin{equation}
\tau = J\dot{\omega} + B\omega + m
\end{equation}

\noindent
The electromechanical equations state that the back emf voltage is proportional to the angular velocity and the motor torque is proportional to the armature current:

\begin{equation}
\begin{gathered}
e = k_{e} \omega \\
\tau = k_{t} i
\end{gathered}
\end{equation}

\noindent
The electrical power consumed and mechanical power consumed will be equal to:

\begin{equation}
\begin{gathered}
p_{\text{e}} = vi \\
p_{\text{m}} = \tau\omega
\end{gathered}
\end{equation}


\subsection{Robot Dynamics}
The robot is modeled as a mass $m$, that is moved by two wheels with radius $r$, each connected directly to a motor.


The rolling friction between the wheels and the surface is equal to:
\begin{equation}
F_{\text{k}} = \mu_{\text{k}}mg
\end{equation}

Therefore the torque applied to the motor due to rolling friction, as it is only present while the robot is moving relative to the surface and the equation becomes:

\begin{equation}
T_{\text{ext}} = rF_{\text{k}} sgn(\omega)
\end{equation}

\noindent
The total mass is equal to the 



\section{Design Considerations}
\label{sec:design_considerations}

This section will shortly explain the main areas considered while designing the battery-less transiently-powered robot

% - Optimize or low power consumption (disable or standby sensors and motor ctrl when not used)
% - Minimal basic functionality (for simple swarm algorithms?) (but no power hungry components ie optical encoders or mouse sensors)
% - Low power communication
% - Navigation

% Extra extra:
% - Tradeoff chargetime and operation time

\begin{enumerate}
	\item \textbf{Power}. 
	The robot should not rely on batteries, alternatively energy can be harvested from ambient sources and stored in a supercapacitor. 
	Energy harvested in a controlled environment should charge the capacitor in under 10 seconds and stored energy should provide at least an operation time of 1 second, allowing the robot to make short controlled movements.
	
	\item \textbf{Small form factor}. 
	By making the robot as small as possible, weight is kept to a minimum reducing the energy required for movement.
	Secondly, by design the robot using low cost off-the-shelf parts, should make it convenient to build collectives of transiently-powered robots.
	
	\item \textbf{Locomotion}.
	Movement will consume the largest amount of energy from the total available energy budget.
	To optimize the distance that can be covered with a single capacitor charge, an efficient locomotion type should be chosen for the movement on flat surfaces.
	
	\item \textbf{Autonomous navigation}.
	During operation the robot will experience a very frequent loss of power. 
	Despite regular power interruption the transiently-powered robot should be able to complete a movement with an acceptable error compared to the same robot being battery powered.
	
	%TODO write about ability to expand to swarming
\end{enumerate}


%\section{Transiently-powered Robot Model}
\label{sec:pre_transient_model}

In this section a simple model will be derived showing the relation between size and weight of the robot, the amount of power that is harvested and stored and how much of this can be translated into linear movement.

\subsection{Modeling Assumptions}

Energy is harvested from an ambient source and stored in a supercapacitor.
To make efficient use of the energy stored in a supercapacitor a regular is required to supply a stable voltage to the connected loads.
In this case the loads are two identical dc motors which each drive wheel. \\
\\ \noindent
The following assumptions will be used to model the transiently powered robot:
\begin{itemize}
	\item The required power by the loads is greater than the incoming power, resulting in repeated power cycling of the robot.
	\item The amount of input power after conversion is constant due to the use of a controlled environment
	\item Since a regulator is used, the voltage in the capacitor will never fall below the operating voltage.
\end{itemize}

%The input power Pin, will be stored in a supercapacitor with capacitance C.


%The regulated output voltage is a lower threshold for the energy that can be used from the capacitor.
%The upper threshold is determined by the maximum voltage rating of the supercapacitor.
%Lowering the output voltage allows for more energy to be used from the supercapacitor, and also lowers the overall power consumption of individual components.
%The energy stored in supercapacitor is a function of the capacitance and the threshold voltage difference, being equal to:








% 1 Incomming power V * I which scales with solar panel size

% 2 Maximum power point tracking (switchmode boost converter)

% 3 Stored in non-ideal supercapacitor with capcitiy C and a parrallel resistance Rleak and series resistance (ESR, typically small but not neglectable?)

% 123 determine chargetime

% Buck converter losses 

% Power consumed from source = Pcons = Ploss + Pweels

% Power P(t)  = F * v = T x omega

\subsection{Motor Dynamics}

The electrical equivalent circuit of a brushed dc motor is shown in Figure \ref{fig:pre_model_dc}, where $v$ is the voltage applied to the motor, $i$ the armature current, $R$ the armature resistance, $L$ the armature inductance, $e$ the back EMF voltage, $\tau$ the torque produced by the motor, $\omega$ the angular velocity of the rotor, $J$ is the moment of inertia of the rotor, $B$ is the viscous friction coefficient of the motor bearings and $m$ the external applied torque.


\begin{figure}[h!]
	\centering
	\begin{circuitikz}
		%\draw [help lines] (-1,-2) grid (12,5);
		
		% electrical equivalent circuit
		%\draw (0,0) to[V, v_=$v$] (0,3);
		\draw (0,3) node[ocirc] {}; % ,label=left:+
		\draw (0,3) to[R, i>^=$i$, l=$R$] (3,3);
		\draw (3,3) to[L, l=$L$] (4,3);
		
		\draw (0,2.25) node {$+$};
		\draw (0,1.5) node {$v$};
		\draw (0,0.75) node {$-$};
		
		\draw (4,3) -- (5,3);
		\draw (5,3) -- (5,2);
		%\draw (5,1.5) node[elmech](motor){M};
		\draw (5,1) -- (5,0);
		
		\draw (4.25,2.25) node {$+$};
		\draw (4.25,1.5) node {$e$};
		\draw (4.25,0.75) node {$-$};
		
		\draw (0,0) -- (5,0);
		\draw (0,0) node[ocirc] {}; 
		
		% motor
		\draw[fill=white] (4.85,0.85) rectangle (5.15,2.15);
		\draw[fill=white] (5,1.5) ellipse (.45 and .45);
		
		
		% shaft drive -> transmission
		\draw[fill=black] (5.45,1.45) rectangle (7.0,1.55);
		
		% momentum arrow of drive -> transmission
		\draw[line width=0.7pt,<-] (5.8,1) arc (-30:30:1);
		
		% moment of inertia
		\draw[fill=white] (7.5,1.59)
		ellipse (.15 and 0.4);
		\draw[fill=white, color=white] (6.9, 1.99)
		rectangle (8.49, 1.19);
		\draw (6.8,1.59) ellipse (.15 and 0.4);
		\draw (6.8,1.99) -- (7.5,1.99);
		\draw (6.8,1.19) -- (7.5,1.19);
		
		% momentum arrow (left hand side of brake shoe)
		\draw[line width=0.7pt,->] (8.05,1.1) arc (-30:30:1);
		
		% descriptions inside graphic
		\draw (5.85,2.2) node {$\omega_A, M_A$};
		\draw (7.25,1.61) node {$J$};
		\draw (8.05,2.32) node {$M_R$};
		
	\end{circuitikz}
	\caption{Brushed DC motor system model.}
	\label{fig:pre_model_dc}
\end{figure}

\noindent
Using Kirchhoff's voltage law the electrical dynamics of a dc motor can be described as
\begin{equation}
\label{eq:kirchhoff}
v = Ri + L \dot{i} + e
\end{equation}

\noindent
From Newton's second law follows that the mechanical dynamics of a motor can be described as
\begin{equation}
\label{eq:newton}
\tau = J\dot{\omega} + B\omega + m
\end{equation}

\noindent
The electromechanical equations state that the back EMF voltage is proportional to the angular velocity and the motor torque is proportional to the armature current

\begin{equation}
\label{eq:electomechanical}
\begin{gathered}
e = k_{e} \omega \\
\tau = k_{t} i
\end{gathered}
\end{equation}

\noindent
where $k_{e}$ is the back emf constant of the motor and $k_{i}$ the torque constant of the motor.
The electrical power consumed and mechanical power consumed will be equal to

\begin{equation}
\begin{gathered}
p_{\text{e}} = vi \\
p_{\text{m}} = \tau\omega
\end{gathered}
\end{equation}

\noindent
Rewriting equation \ref{eq:kirchhoff}, \ref{eq:newton} and \ref{eq:electomechanical} and appling the Lalace transa transfer function from $v$ to $\omega$ can be obtained, assuming $m$ = 0.

\begin{equation}
\frac{\Omega(s)}{V(s)} = \frac{k_{i}}{(Ls + R)(Js + B) + k_{\omega}k_{i}} 
\end{equation}

where $V(s)$ and $\Omega$ are the Lapace transformations from $v$ and $\omega$ respectively.


\subsection{Robot Dynamics}
The robot is modeled as a mass $m$, that is moved by two wheels with radius $r$, each connected directly to a motor.


The rolling friction between the wheels and the surface is equal to:
\begin{equation}
F_{\text{k}} = \mu_{\text{k}}mg
\end{equation}

Therefore the torque applied to the motor due to rolling friction, as it is only present while the robot is moving relative to the surface and the equation becomes:

\begin{equation}
T_{\text{ext}} = rF_{\text{k}} sgn(\omega)
\end{equation}

\noindent
The total mass is equal to the 



\chapter{Design and Implementation}
\label{chp:design_and_implementation}

The hardware design of the transiently-powered robot is presented in Section \ref{sec:dai_hardware_design}.
In Section \ref{sec:dai_control_design} the controller is explained that allows the robot to perform controlled movements without external feedback.
Finally, Section \ref{sec:dai_software_implementation} introduces the software implementation that enables the robot to execute a movement despite regular power interrupts. 

\section{Hardware Design}
\label{sec:dai_hardware_design}
The first step in the hardware design is to evaluate what components are required for the robot to have basic navigation capabilities. 
Commercially available low power components have been evaluated, where the main criteria was a low minimal supply voltage of 2.0\,V to function with a system voltage of 2.2\,V.
In this section each part of the robot is explained in more detail.
A complete overview of the robot is shown in Figure \ref{fig:robot_overview} and a complete assembled robot with WISP and solar panel is shown in Figure \ref{fig:robot_picture}.

\vspace{1em}
\begin{figure}[h!]
	\centering
	\includegraphics[width=0.9\textwidth]{pics/schematic_robot_v2.png}
	\caption{Schematic overview of the transiently-powered robot.}
	\label{fig:robot_overview}
\end{figure}

\begin{figure}[h!]
	\centering
	\includegraphics[width=0.9\textwidth]{pics/tp_robot2.png}
	\caption{The complete robot with WISP and solar panel.}
	\label{fig:robot_picture}
\end{figure}

\subsection{Energy Harvesting and Storage}
Energy is harvested from light using two IXYS SLMD121H04L-ND solar cells~\cite{ixolar_slmd121h04l_2017} in parallel, selected based on experimental results given in Section \ref{sec:pre_energy_source_selection}.
The solar cells are connected to a Texas Instruments BQ25570 energy harvester~\cite{bq25570_2017} which stores the harvested energy in a 22\,mF - 4.5\,V supercapactor from AVX~\cite{avx_bestcap_2017}.

\subsection{Computation}
\label{sec:dai_computation}

The robot is designed around a WISP5 \cite{wisp5_wiki_2017}, a battery-free platform for low power sensing, computation and communication.
This platform has the ability to communicate with RFID readers and is powered by the carrier signal emitted by the reader.
However, the communication range and the power that can be harvested is limited. 
Only the MCU from the WISP is currently being utilized: a Texas Instruments MSP430FR5969 ultra low power microcontroller.
This MCU can operate at 16\,MHz and features 64\,KB FRAM, 2\,KB SRAM and 40 IO-ports \cite{msp430fr5969_2017}.

%Therefore communication is currently not implemented and a different energy source for harvesting is used as described in Section \ref{sec:pre_energy_harvesting_storage}.

\subsection{Sensing}
\label{sec:dai_sensing}

The robot has access to basic sensors which can be interfaced trough I2C.
For detecting obstacles in front of the robot, a Maxim Integrated MAX44000 proximity sensor~\cite{max44000_2017} was added to the robot facing forward.
The sensor switches an IR led at high frequency to reduce the power consumption.
Because the sensor is based around a photo-diode it can be used to measure the amount of ambient light as well.
To allow for local motion feedback, the robot has a Bosch Sensortec BMG250~\cite{bosch_bmg250_2017} low power triaxial gyroscope to measure yaw-rate.

\subsection{Locomotion}

Sub-micro plastic planetary gearmotors from Precision Microdrives~\cite{gearmotor_206-110_2017} are chosen to provide locomotion to the robot, as described in Section \ref{sec:pre_locomotion_selection}.
Two motors are mounted diagonally opposite from each other in a 3D-printed frame making the robot as compact as possible, and the differential drive configuration allows steering.
Small plastic wheels with rubber tires are mounted directly on each of the motor shafts.
Behind these two motors a free running caster wheel is mounted to the frame, acting as a third support point for the robot.

\subsection{Motor Control}
\label{sec:dai_motor_control}

The speed and the current consumption of each motor can be controlled individually using PWM.
The MCU can use the H-bridge to enable, disable and control the direction of rotation of each individual motor.
MCU output ports are limited in the amount of current that they can supply.
MOSFETs inside a Texas Instruments DRV8836 dual H-bridge~\cite{drv8836_2017} allow efficient regulation of larger currents to the motors.

\subsection{Integration}
Now that all the parts have been chosen, they can be connected together to form the robot.
A Printed Circuit Board (PCB) has been designed using EAGLE PCB design and schematic software~\cite{eagle_pcb_2017}.
The PCB increases stability, eases connection and reduces of the total weight of the robot.
Additionally, a PCB was required because of use of small size ICs (no lead packaging).
A detailed schematic of the PCB can be found in Appendix \ref{app:schematic_robot}.

The size of the PCB is 30$\times$30\,mm, and an overview of both the top and bottom side of the PCB is shown in Figure \ref{fig:pcb_robot}.
The PCB contains headers to connect the solar panel and headers to connect all the required pins from the WISP.
An additional header is available to connect a battery for testing purposes.
All the large components are mounted externally: the solar panel, WISP and the motors.
%Most reference designs are extendable but this adds complexity, while weight and size are the main constraints in the design of this robot.

%The top of the PCB contains all the surface mount components that are soldered to the PCB using a reflow oven. 
%When the boards come out of the oven, the headers, motors and supercapacitor are soldered to the boards by hand.

\begin{figure}[h!]
	\centering
	\begin{subfigure}[b]{0.45\textwidth}
		\includegraphics[width=\textwidth]{pics/pcb_front.jpg}
		\caption{Top side of the PCB}
		\label{fig:pcb_robot_front}
	\end{subfigure}
	\qquad
	\begin{subfigure}[b]{0.45\textwidth}
		\includegraphics[width=\textwidth]{pics/pcb_back.jpg}
		\caption{Bottom side of the PCB}
		\label{fig:pcb_robot_back}
	\end{subfigure}
	\caption{The PCB designed for the robot, on the top side of the PCB the ICs are marked, harvester (U1), gyroscope (U2) and H-bridge (U3). The bottom side only contains the supercapacitor.}
	\label{fig:pcb_robot}
\end{figure}

\subsection{Energy Expenditure}

The average current consumption by each component is measured with a Monsoon Power Monitor~\cite{monsoon_powermonitor_2017}.
The measurement is performed as follows: first a two second current trace was recorded of the current consumed by the MCU on the WISP.
The MCU was then used to enable each component individually, followed by a two second current measurement with the enabled component.
The average of each current trace is calculated, and the current consumed by the microcontroller subtracted from the current consumed by each component.
The measurements results are provided in Table \ref{tab:avg_cur_comp}.

% Make a overview of the cost to build a robot

\begin{table}[t]
	\centering
	\begin{threeparttable}
		\caption{Average consumed current for each individual component on the PCB at 2.2\,V.}
		\label{tab:avg_cur_comp}
		\begin{tabular}{|l|l|} 
			\hline
			Part & Active Current \\
			\hline\hline
			Proximity sensor & 119 \textmu A \\
			Gyroscope & 848 \textmu A\\	
			Microcontroller @ 8MHz & 522 \textmu A\\
			H-bridge & 349 \textmu A \\
			Two DC motors\textsuperscript{1} & 27--50 mA  \\
			\hline \hline
			Total & 29--52 mA \\
			\hline
		\end{tabular}
		\begin{tablenotes}
		\small
		\item [1] Current consumed varies per motor and motor speed
		\end{tablenotes}
	\end{threeparttable}
\end{table}

\subsection{Costs}

Using of off-the-shelf components that are readily available allows to build multiples of this robot with ease.
%The cost per individual is important when these robots are used in collectives.
Table \ref{tab:cost_robot} shows the total price per robot, assuming a minimal fabrication quantity of 20.
This overview is compiled by querying the component prices of different suppliers, Farnell, Digikey, Mouser, Pololu, and OshPark.
The price for the PCB does not include the price of assembling the PCB, as it is currently done by hand.
Secondly, the cost of a WISP/MCU is currently not included in the price.
From this table can be concluded that the cost of building a transiently powered robot is comparable to the cost of other reference small robotic platforms, as seen in Table \ref{tab:comparison_robot_platforms}.


\begin{table}[t]
	\centering
	\caption{Cost of parts for one robot}
	\label{tab:cost_robot}
	\begin{tabular}{|l|l|} 
		\hline
		Part & Price (\euro) \\
		\hline\hline
		Solar panel & 9,60\\
		Supercapacitor & 7,43\\
		Harvester & 5,48 \\
		Proximity sensor & 3,98 \\
		Gyroscope & 2,81\\	
		H-bridge & 1,47 \\
		Two DC motors & 18,51 \\
		Wheels & 3,08\\
		PCB & 2,50 \\
		Passive SMD & 4,53\\
		\hline \hline
		Total & 59,39 \\
		\hline
	\end{tabular}
\end{table}


\section{Motion Control Design}
\label{sec:dai_control_design}
In order for the robot to make controlled movements, it needs to have a local heading feedback method.
Since the buck converter supplies a constant voltage to the motors, voltage is eliminated as a factor in determining the motor speed.
By making the assumption that the robot only travels on a flat surfaces, the steady state speed is considered ``constant".

\subsection{PWM Frequency for Linear Motion Control}
\label{sec:cd_pwm_frequency}

PWM is used to control the speed of the motors, as briefly addressed in Section \ref{sec:pre_dc_motor_locomotion}.
The maximum frequency that still allows linear speed control is dependent on the electrical characteristics of the motor.
When the motor is at rest, its equivalent circuit consists of a resistance (R) and inductance (L) in series.
If a voltage is applied to the motor, the rate at which the current rises is limited by the inductance. 
All RL circuits have a time constant: $\tau = L / R$ and the current is considered to have reached its maximum steady state at $5\tau$~\cite{pmw_linear_motion_2017}. 
The motors used for the robot have a typical resistance of R = 14.5\,$\Omega$ and an inductance of L = 70\,$\mu$H~\cite{gearmotor_206-110_2017}.
Therefore, the minimum pulse width should be equal to

\begin{equation}
T_{\min} = 5 \frac{L}{R} = 24.14\,\mu\text{s}.
\end{equation}

\noindent
If a minimum duty cycle $D_{\min}$ of 5\% assumed, than the maximum PWM frequency becomes

\begin{equation}
f_{\max} = \frac{D_{\min}}{T_{\min}}\times 100 = 2071.25\,\text{Hz}.
\end{equation}

\noindent
The PWM frequency is set to 2\,kHz and from Table \ref{tab:duty_cycle} can be seen that the minimum duty cycle is always above 5\%.

\begin{table}[t]
	\centering
	\caption{Minimum and maximum duty cycle for different robots.}
	\label{tab:duty_cycle}
	\begin{tabular}{|l||l|l|l|} 
		\hline
						          & Robot 1 & Robot 2 & Robot 3 \\
		\hline \hline
 		Min duty cycle left (\%)  & 16      & 14      & 18      \\
		Min duty cycle right (\%) & 13      & 22      & 16      \\
		Max duty cycle (\%)       & 30      & 36      & 34	    \\
		\hline
	\end{tabular}
\end{table}

\subsubsection{Minimum Duty Cycle}

The minimum duty cycle is determined by the torque that is required for the motors to be able to overcome the static friction between the wheels and a surface the robot is moving on.
Each motor is physically different and the friction in the gearbox can variate as well, which results in different output speeds per motor.
Since the robot uses two motors in differential drive configuration, a minimum duty cycle has to be found for each motor.
This is accomplished by setting a duty cycle at which both motors are rotating and slowly backing it down until one or both motors stop turning, the minimum duty cycle for each motor can be seen in Table \ref{tab:duty_cycle}.
The minimum duty cycle, allowing each wheel to rotate, is stored as a calibration value.
If each motor is supplied their minimum duty cycle does not imply that the motors rotate at the same speed, i.e. that the robot makes a straight movement.

\subsubsection{Maximum Duty Cycle}

The maximum duty cycle is bounded by the amount of current that the buck converter and bulk capacitor can supply, as shortly discussed in Section \ref{sec:pre_dc_motor_locomotion}.
Lowering the duty cycle reduces both the maximum current peak and the steady state current, which is used to reduce the motor start current demand.
The maximum duty cycle can be found by increasing the minimum duty cycle of each motor with the same value until the robot is unable to start a movement.
The last working value is stored as a calibration value to limit the duty cycle, and the average maximum duty cycle is given in Table \ref{tab:duty_cycle}.
The maximum duty cycle values are all below the calculated free running maximum duty cycle, as previously determined in Section \ref{sec:pre_dc_motor_locomotion}


\subsection{Closed loop feedback for controlled movements}

%The robot uses two physically different motors in differential drive configuration which are mounted in a non-symmetrical way.
Open loop movement using calibrated motor values has been used in previous work \cite{legoc_uist_2016}, but it can be time consuming.
Furthermore, any small disturbance can throw the robot off course.
Controlled movements can be achieved by using closed loop feedback, where the heading is used to update the motor control values.
The robots relative change in heading, i.e. horizontal angular velocity, can be obtained from the gyroscope and corresponds to the yaw-rate.

\subsubsection{PID Controller}

% -Why pid for straight movements and not a simple p controller?
% --Fast reaction on disturbances without osccilation??
%TODO -Write about bounding the pid output, because otherwise the motors of the robot could stall, if the motor setpoint is to high

Closed loop feedback is achieved by use of a Proportional Integral Derivative (PID) controller.
The input of the PID controller is the yaw-rate from the gyroscope.
The controller periodically tries to reduce the yaw error as

\begin{equation}
	e(t) = \psi_{\text{target}} - \psi(t),
\end{equation}

\noindent
where $\psi_{\text{target}}$ is the yaw-rate target and $\psi(t)$ the yaw-rate obtained by the gyroscope.
The PID controller adjusts its output, and corrects the speed of each motor in opposite direction as

\begin{equation}
u(t) = K_{\text{p}}e(t) + K_{\text{i}} \int_{0}^{t}e(\tau)d\tau + K_{\text{d}}\frac{d}{dt}e(t),
\end{equation}

\noindent
where $K_{p}$, $K_{i}$ and $K_{d}$ are the tunable gains from the PID controller.
Using the gains, the controller is continuously adjusting the motor speed in order to reduce the error to zero.

\subsubsection{Controlled movements}

The robot is able to execute two distinct movement: straight and curved movements.
When the robot executes a controlled straight movement, any movement perpendicular to the robots heading direction is undesired.
The yaw-rate target is set to zero, forcing the PID controller to keep the robot straight.
For curved movements however, a desired yaw-rate set point needs to be specified.
The yaw-rate set point can for example be determined from the radius of the circle that the robot needs to turn and a calibrated speed.

\begin{table}[t]
	\centering
	\caption{Ziegler-Nicholos PID gain estimator chart~\cite{franklin_feedback_2015}.}
	\label{tab:gain_chart}
	\begin{tabular}{|l|l|l|} 
		\hline
		$K_{\text{p}}$ & $T_{\text{i}}$ & $T_{\text{d}}$ \\
		\hline \hline
		0.6$K_{\text{u}}$ & $T_{\text{u}}/2$ & $T_{\text{u}}/8$ \\
		\hline
	\end{tabular}
\end{table}

\subsubsection{PID tuning using Ziegler-Nichols method}

Tuning can be done by a trail and error approach, but a faster way of tuning is to use the closed loop Ziegler-Nichols method~\cite{franklin_feedback_2015}.

The method evaluates the amplitude and frequency of observed oscillations in the system by adjusting the tunable gains.
Initially, the integral gain and the derivative gain, $K_{\text{i}}$ and $K_{\text{d}}$ respectively, are set to zero.
Then, the proportional gain $K_{p}$ is increased from zero until sustained oscillation occurs, which corresponds to the ultimate gain $K_{\text{u}}$.
The ultimate period $T_{\text{u}}$ is equal to the corresponding period and should be measured at zero crossings of the oscillation.
The found ultimate gain and period can be used to determine the PID gains using Table \ref{tab:gain_chart}.

\subsubsection{Experimental ultimate gain and period determination}	

To find the ultimate gain and period, a robot is programmed to execute a two second straight movement.
The yaw-rate data is stored in non-volatile memory in order to retrieve it using a programmer, after the robot has performed the movement.
In Figure \ref{fig:ultimate_gain} two second yaw-rate measurement traces are shown for several proportional gains.
With a proportional gain of 0.13 the robot shows roughly constant oscillation.
On the other hand, for a gain of 0.14 the robot shows unstable behavior due to a small disturbance after one second, resulting in oscillations that keep increasing in amplitude.
The ultimate period is determined to be equal to $T_{u} = 0.2$, and is used together with the ultimate gain to determine the tuning parameters from the gain chart in Table \ref{tab:gain_chart}.
The integral gain and derivative gain can now be determined to be equal to $K_{\text{i}} = K_{\text{p}} / T_{\text{i}}$ and $K_{\text{d}}  = K_{\text{p}}T_{\text{d}}$ respectively.
Figure \ref{fig:gain_tuning} shows how the oscillations are removed by setting the tunable gains according to the closed loop Ziegler-Nichols method.
The tuning process was speeded up by setting the minimum motor duty cycle as it made the robot a lot more responsive, as described in Section \ref{sec:cd_pwm_frequency}.

\begin{figure}
	\begin{subfigure}[b]{0.5\textwidth}
		\includegraphics[width=\textwidth]{pics/straight_ku.png}
		\caption{Determining the ultimate gain}
		\label{fig:ultimate_gain}
	\end{subfigure}
	\begin{subfigure}[b]{0.5\textwidth}
		\includegraphics[width=\textwidth]{pics/straight_ku_with_tu.png}
		\caption{Result of applying the gains}
		\label{fig:gain_tuning}
	\end{subfigure}
	\caption{Tuning the PID controller using the closed loop Ziegler-Nichols method, first the ultimate gain is determined to be equal to 0.13. Then the ultimate period is determined equal to 0.2 and using Table \ref{tab:gain_chart} the tunable gains are determined.}
\end{figure}

%\subsubsection{Controlled turns}

%The control loop for controlled turning uses the angle, which is obtained by integrating the yaw-rate sensor data from the gyroscope.
%A P controller is used to rotate the robot to the desired angle, the proportional gain is directly influencing turn speed of the robot.
%The motor control values are set to run the motors opposite directions and are equal to the output of the P controller.
%These values keep decreasing until the robot rotates to the desired angle.


\section{Software Implementation}
\label{sec:dai_software_implementation}

The software implementation that allows the robot to perform controlled movements is divided in three parts, the main program, a control loop and motor control.
In the main program a set of movements can be defined for the robot to be executed.
The robot can be controlled using three movement commands: (1) straight trajectories, (2) curved right turns and (3) curved left turns.
For each movement an additional movement target needs to be defined that specifies the duration, i.e. when the control loop considers the movement to be finished.

\subsection{Control loop}

When the control loop receives a movement command, the yaw-rate set point, movement target and motor duty cycle target are set accordingly.
A timer running at a frequency of a 100\,Hz, is used to periodically call an interrupt service routine (ISR) and execute the control loop.
Using an ISR has the benefit of a constant sample time, which simplifies the PID loop, because the integration and differentiation time are also constant and known in advance.
When the ISR is triggered, first a yaw-rate sample is requested from the gyroscope, as seen in Figure \ref{fig:flowchart_code}.
Based on the current movement command, an evaluation is performed to determine if the movement target is reached.
If this is not the case, the yaw-rate is supplied to the PID controller, which in turn updates the motor values.
The output value produced by the PID controller is subtracted from the left motor value and added to the right motor value, in order to keep the average speed approximately the same.

\subsubsection{Movement target}

For straight movements the movement target is equal to a predetermined time, i.e. number of control loop timer trigger events.
Curved movements use an angle movement target.
By integrating the angular velocity an angle estimate is obtained, which is used to verify if the provided angle movement target is reached within a margin of two degrees.
The loop exits automatically when the required movement target is reached.

\begin{figure}[ht!]
	\centering
	\includegraphics[width=0.65\textwidth]{pics/flowchart_code.png}
	\caption{The control loop that periodically updates the motor values based on the yaw-rate obtained from the gyroscope. The loop exits automatically if the movement target is reached.}
	\label{fig:flowchart_code}
\end{figure}

\subsection{Motor Control}
% -Write how the pwm control signals are generated for the H-bridge.
The motors are controlled using PWM signals generated by a second timer, which runs at the predetermined frequency of 2\,kHz.
The timer is able to directly control the four IO-ports connected to the H-bridge, eliminating the overhead of an ISR.
The H-bridge is configured such that two IO-ports directly control a motor.
If one of the ports is enabled the motor rotates forwards and if the other port is enabled the motor rotate backwards.
The control loop updates the compare registers corresponding to each of the ports.
The minimum duty cycle value is added to each motor value and is bounded by the maximum duty cycle.
When the timer reaches a value that corresponds to the value stored in one of the compare registers, the connected port is toggled automatically.

%\subsection{Controlled Turns}
%The set point is assumed to be reached when the angle is within two degrees of the target, in this case the loop stops automatically.
%To allow enough precision to measure if the target is within these two degrees, the timer which executes the control loop was set to run at 100\,Hz.
%Secondly, the proportional gain should not be set to low because the robot might not able to reach the target but also not to high as it can overshoot the target.

\subsection{Persistent Movement}
The transiently-powered robot can make one movement or a series of movements, which probably requires multiple power cycles to complete.
To be able to finish a movement and not reset, i.e. redo the same movement, a simple checkpointing method is used to save the progress across power cycles.
A persistent counter registers the progress in the set of movements.
Every control loop iteration the persistent variable that captures the progress towards the movement target is updated, and depending on the movement can be a time or angle.
The right and left motor speed tuned by the PID controller are not saved and restored after a power interrupt.
Due to the startup phase of the motors, previously tuned motor speeds are suspected to be invalid and cannot be used after a power interrupt. 

\subsection{Extendability}
The current software implementation consists of 700 lines of MSP430 specific C code, that takes up 14\% of the SRAM and 14\% of the FRAM.
The control loop is implemented using an ISR leaving the main processor available for other computation tasks.
The backscatter communication channel on the WISP is currently not implemented, but can potentially be used to provide communication with a global host. 
Two of the five available timers are used for the control loop and the motor control, while the other three timers are used by the WISP software communication stack.

\chapter{Evaluation} 

\section{Controlled Movements}
\label{sec:controlled_movements}

% Comparison deadreckoning accuracy of battery powered robot with solar powered robot

% Try at least one more supercapacitor: 10mF
% Study reducing the frequency of power interrupts ie smaller energy buffer.
% How does this effect the accuracy of locomotion?
% Can the frequency of power interrupts be related to the 

% Video of robot movement

In this section the accuracy of movement of the battery-less robot and its battery powered counterpart will be compared.

\subsection{Experimental setup}

To be able to compare the accuracy of the robot while it is exposed to increasingly smaller power cycles, a variety of movements is recorded using a overhead camera.
A camera stand with a Nikkon D610 DSLR camera is positioned on a tabletop and in the camera's view the corners of a square of 80$\times$80\,cm are indicated with a black marker, see Figure \ref{fig:movement_setup}.
This square is later used as a reference to convert the robots movement from pixels to cm.
Three different movements are compared, while the robot is moving straight movement of 75\,cm, a circle with radius of 30\,cm and a square of 50$\times$50\,cm.

\begin{figure}
	\centering
	\begin{subfigure}[b]{0.45\textwidth}
		\includegraphics[width=\textwidth]{pics/movement_setup.jpg}
		\caption{Camera setup}
		\label{fig:movement_setup}
	\end{subfigure}
	\quad
	\begin{subfigure}[b]{0.45\textwidth}
		\includegraphics[width=\textwidth]{pics/movement_example.png}
		\caption{Tracking with OpenCV}
		\label{fig:movement_example}
	\end{subfigure}
	\caption{Experimental setup to record the robots movement}
\end{figure}

\subsubsection{Tracking the Movement}

%The robot is programmed to preform the movement at a desired PWM target ie at a specified speed, and optional power interrupt period.
Before the robot executes the movement a green led is enabled on top of the robot.
This green dot will be the reference point that the tracking software will try to follow.
The camera is used to record the movement which then is analyzed using Python and OpenCV 3.2.
An example of a tracked movement can be seen in Figure \ref{fig:movement_example}.

\subsubsection{Target Speed Setting}
To evaluate the influence of speed each movement is executed at three different speed target settings: 40\%, 65\% and 90\% of the maximum duty cycle.
Choosing a target higher than 90\% is undesired while the controller needs some room to adjust the speed up and down to control the robots heading.

\subsubsection{Power Interrupts}

To evaluate different power cycle periods without changing the capacitor size or the minimum and maximum voltage thresholds set using resistors, power interrupts are generated artificially.
During the experiment the robot will be powered from a battery and power interrupts, i.e the capacitor running out of energy are created artificially using a timer that resets the MCU.
The MSP430FR5969 has the functionality to enable a brownout reset trough software which is used to simulate the event of the supply voltage dropping below the required operating voltage~\cite{msp430fr_family_guide_2017}.
A timer is used to generate the power interrupt after a predefined period.

\subsubsection{Power Interrupt Period}

With the selected capacitor of 22\,mF the robot can operate around 1 second. 
Choosing a period of 0.25\,s showed uncontrolled behavior, while the robots control loop was not able to stabilize the movement before a power interrupt occurs.
To find the minimal power interrupt period that is required to control the robot, the robot is programmed to preform a four second straight movement.
The movement is recorded for each target speed and the power interrupt periods of 0.4\,s, 0.3\,s and 0.2\,s are evaluated.

The results in Figure \ref{fig:decreasing_power_period} show that a interrupt period of 0.2\,s results in uncontrolled behavior, while the robot drifts to the side of the weakest motor.
Increasing the target speed shows that also a higher interrupt period period is required to control the movement. 
The power interrupt periods evaluated in this experiment were set to values above and equal to 0.5\,s: 1.25\,s, 1\,s, 0.75\,s and 0.5\,s.


\begin{figure}
	\begin{subfigure}[b]{0.32\textwidth}
		\includegraphics[width=\textwidth]{pics/figure_40.png}
		\caption{Target 40\%}
		\label{fig:target_40}
	\end{subfigure}
	\begin{subfigure}[b]{0.32\textwidth}
		\includegraphics[width=\textwidth]{pics/figure_65.png}
		\caption{Target 65\%}
		\label{fig:target_65}
	\end{subfigure}
	\begin{subfigure}[b]{0.32\textwidth}
		\includegraphics[width=\textwidth]{pics/figure_90.png}
		\caption{Target 90\%}
		\label{fig:target_90}
	\end{subfigure}
	\caption{Accuracy of movement with decreasing the power interrupt period.}
	\label{fig:decreasing_power_period}
\end{figure}

\subsubsection{Velocity Calibration}
Another thing that can be observed from the results in Figure \ref{fig:decreasing_power_period}, is that the distance covered by the robot deceases by deceasing the target and/or power interrupt period.
Without any sensors or external feedback that can determine the robots speed, it is difficult to determine the distance that the robot has traveled.
The average speed is estimated for each speed target and power interrupt period.
This is achieved by first determining the time that the robot requires to move approximately 150\,cm for each target without power interrupts.
When the robot experiences power interrupts the average velocity of an active period becomes lower due to frequent acceleration from a standstill.
With power interrupts the runtime is increased to make the robot travel roughly the same distance.
Finally, the average of five complete movement measurements is computed and divided by the commanded runtime of the robot to acquire an average speed for each combination, as seen from Table \ref{tab:val_calib}.


\begin{table}[t]
	\centering
	\small
	\caption{Calibrated velocity}
	\label{tab:val_calib}
	\begin{tabular}{|l|l|l|l|l|l|}
		\hline
		Target (\%) & No int & 1.25\,s & 1.0\,s & 0.75\,s & 0.5\,s \\
		\hline \hline
		 40 & 18.9\,m/s & 18.2\,cm/s & 18.0\,cm/s & 17.1\,cm/s & 15.9\,cm/s \\
	     65 & 24.0\,m/s & 22.2\,cm/s & 21.3\,cm/s & 20.8\,cm/s & 18.8\,cm/s \\
		 90 & 28.8\,m/s & 26.4\,cm/s & 25.7\,cm/s & 24.5\,cm/s & 22.7\,cm/s \\
		\hline
	\end{tabular}
\end{table}

\subsection{Straight Movements}

Using the calibrated velocities and selected power interrupt periods, the robot is commanded to preform a straight movement of 75\,cm.
Each combination is recorded multiple times and the motion data is extracted from the video using Python and OpenCV.
The results in Figure \ref{fig:straight_movements} show that the robot roughly moves the commanded 75\,cm.

\begin{figure}
	\centering
	\begin{subfigure}[b]{0.32\textwidth}
		\includegraphics[width=\textwidth]{pics/straight_40.png}
		\caption{Target 40\%}
		\label{fig:stra_exp1}
	\end{subfigure}
	\begin{subfigure}[b]{0.32\textwidth}
		\includegraphics[width=\textwidth]{pics/straight_65.png}
		\caption{Target 65\%}
		\label{fig:stra_exp2}
	\end{subfigure}
	\begin{subfigure}[b]{0.32\textwidth}
		\includegraphics[width=\textwidth]{pics/straight_90.png}
		\caption{Target 90\%}
		\label{fig:stra_exp3}
	\end{subfigure}
	\caption{Straight movements, the black horizontal line shows the 75\,cm endpoint}
	\label{fig:straight_movements}
\end{figure}

\subsubsection{Movement Accuracy Metrics}
NO EXTERNAL FEEDBACK

Even though the robot is positioned carefully in the same start position, it can happen that the robot is not positioned perpendicular the reference frame of 80$\times$80\cm, i.e. the robot will start moving on a angle.
Therefore 

\begin{itemize}
	\item average curvature of straight line movement

	\item for both straight and curved movements: length of the total movement and  standard deviation in length of movement
\end{itemize}


\begin{table}[t]
	\centering
	\caption{The Euclidean distance between the recorded data and a straight line between the begin and endpoint of the data.}
	\label{tab:straight_results}
	\begin{tabular}{|l|l||l|l|l|}
		\hline
		Target (\%) & Interrupt (s) & Max (cm) & Mean (cm) & Std (cm)\\
		\hline \hline
		\multirow{5}{*}{40} & No int & 0.86 & 0.36 & 0.16 \\
		& 1.25 & 0.67 & 0.34 & 0.12 \\
		& 1.00 & 1.35 & 0.56 & 0.27 \\
		& 0.75 & 1.04 & 0.60 & 0.27 \\
		& 0.50 & 0.86 & 0.44 & 0.22 \\
		\hline
		\multirow{5}{*}{65} & No int & 0.52 & 0.25 & 0.12 \\
		& 1.25 & 0.72 & 0.31 & 0.14 \\
		& 1.00 & 0.91 & 0.44 & 0.21 \\
		& 0.75 & 1.44 & 0.80 & 0.35 \\
		& 0.50 & 2.15 & 1.17 & 0.54 \\
		\hline
		\multirow{5}{*}{90} & No int & 0.54 & 0.28 & 0.13 \\
		& 1.25 & 0.90 & 0.42 & 0.17 \\
		& 1.00 & 1.76 & 0.95 & 0.42 \\
		& 0.75 & 1.88 & 0.88 & 0.37 \\
		& 0.50 & 2.18 & 1.09 & 0.61 \\
		\hline
	\end{tabular}
\end{table}

\subsection{Circular Movements Results}

The robot is commanded make a circle with radius of 30\,cm
\subsubsection{Movement Accuracy Metrics}



\begin{table}[t]
	\centering
	\caption{The mean and standard deviation of the Euclidean distance from the fitted circle.}
	\label{tab:circular_results}
	\begin{tabular}{|l|l||l|l|l|}
		\hline
		Target (\%) & Interrupt (s) & Radius (cm) & Mean (cm) & Std (cm)\\
		\hline \hline
		\multirow{5}{*}{40} & No int & 34 & 1.3 & 0.67 \\
		& 1.25 & 32 & 1.29 & 0.69 \\
		& 1.00 & 32 & 1.52 & 0.81 \\
		& 0.75 & 33 & 1.21 & 0.82 \\
		& 0.50 & 32 & 1.85 & 1.30 \\
		\hline
		\multirow{5}{*}{65} & No int & 34 & 1.2 & 0.67 \\
		& 1.25 & 32 & 1.01 & 0.57 \\
		& 1.00 & 32 & 1.42 & 0.92 \\
		& 0.75 & 33 & 1.11 & 0.73 \\
		& 0.50 & 32 & 0.79 & 0.34 \\
		\hline
		\multirow{5}{*}{90} & No int & 30 & 0.9 & 0.45 \\
		& 1.25 & 30 & 0.81 & 0.43 \\
		& 1.00 & 31 & 1.06 & 0.65 \\
		& 0.75 & 31 & 0.71 & 0.41 \\
		& 0.50 & 35 & 1.69 & 1.23 \\
		\hline
	\end{tabular}
\end{table}


\begin{figure}[h!]
	\centering
	\begin{subfigure}[b]{0.5\textwidth}
		\includegraphics[width=\textwidth]{pics/circle_40.png}
		\caption{Target 40\%}
		\label{fig:circ_exp1}
	\end{subfigure}
	\begin{subfigure}[b]{0.5\textwidth}
		\includegraphics[width=\textwidth]{pics/circle_65.png}
		\caption{Target 65\%}
		\label{fig:circ_exp2}
	\end{subfigure}
	\begin{subfigure}[b]{0.5\textwidth}
		\includegraphics[width=\textwidth]{pics/circle_90.png}
		\caption{Target 90\%}
		\label{fig:circ_exp3}
	\end{subfigure}
	\caption{Circular movements, black dashed circle is the target}
\end{figure}


\subsection{Conclusion}
%Higher speed + more interrupts results in more drift to the left, because of wheel more in the back or weaker motor?
%How the interrupts are distributed over the distance has a influence or how the last interrupt ends up..
%Lower speed and less interrupts results in a higher accuracy??


%For square: Turn\_right speed should be high enough to turn within 0.5 sec


%Observations:
%pwm50 an interrupt every 0.25 sec is uncontrollable
%Lower speed requires longer time between interrupts
%Higher speed + more interrupts results in more drift to the left, because of wheel more in the back or weaker motor?

%Hoe de interrupts uitkomen op de beweging heeft invloed!!

%Lager snelheid en minder interrupts is hogere precisie??




% CONCLUSIONS AND FUTURE WORK
\chapter{Summary}
\label{chp:summary}

\section{Limitations}

%TODO This is only one limitation (speed accurrracy) but there are more - please list them (lack of more sensors, communciation with the extenal world, size can be even smaller, not managed to develop a swarm).

Section \ref{sec:controlled_movements} shows that the transiently powered robot is able to execute a instructed motion with similar accuracy compared to a battery powered equal.
However, the distance covered by a robot with frequent power interrupts in a certain amount of time is smaller, i.e the average speed is lower due to frequent acceleration from a standstill.
For this reason the robot needs a method to determine it's speed without wheel encoders.
One possibility to achieve this would be to determine motor speed by measuring the Back-EMF produced by the motor, as it is proportional to the motors revolutions per minute. % ~cite{precision_backemf_2017}.
However, the amount of power required to frequently determine the Back-EMF needs to be evaluated carefully.

Another possibility is give external feedback to the robot about its current position or control the robot's movement externally.
To achieve this the backscatter communication channel of the WISP could be used.
% ADD some info about fast downstream communication?

\section{Future Work}

This work focuses on the development of a single transiently powered robot.
A new promising area of research could be to create a swarm of transiently powered robots, and investigate the effect of intermittentcy on the behavior and controllability of this type of swarm.
This research should further investigate the portability of existing swarm algorithms, and if not possible propose new solutions.	

Recently, embedded operating systems~\cite{trenkwalder_iros_2016} and extendable programming~\cite{pinciroli_iros_2016} languages have been created to speed up the development process, removing the need to focus on low level interactions and individual behaviors.
In addition, multiple task and checkpoint based methods have been developed to enable computation across power cycles as described in Section \ref{sec:comp_pc}.
Merging both paradigms could help to speed up development of transiently powered swarms.

% Swarms restricted to light availability 

% different locomotion types?
In order to further significantly reduce the weight of a robot there is a need to move away from DC-motors, because further miniaturizing DC-motors has become a challenge.
Alternative forms of locomotion has been discussed in Section \ref{sec:locomotion}.
Micro size legged robots that make use of piezoelectric actuators have been developed but currently lack any intelligence. REF I-SWARM?

%\subsection{Path planning based on energy availability}

%To capture the optimal amount of solar energy along the way, a map of the expected solar power can be used to compute the optimal path. To distinguish sunny or shaded two methods are proposed in \cite{plonski_tranro_2016}, one being a simple data driven Gaussian Process and the other estimates the geometry of the environment as a latent variable.
%Energy aware path planning is commonly used in combination with mission planning.
%In \cite{kaplan_iros_2016}, an analysis of the solar radiation is used to generate a time-optimized motion plan and power schedule using a cascaded particle swarm optimization algorithm.
%By combining maps of lighting and ground slope a solar-powered robot can be kept illuminated continuously. A connected component analysis is used to plan a optimal route on traversable slopes, as described by \cite{otten_icra_2015}.

\section{Conclusions}
% TODO CONCLUSIONS

% BIBLIOGRAPHY
%#define SORTED 1
%\bibliographystyle{../bib/latex8}
%\bibliography{../bib/bibtex}

\bibliographystyle{IEEEtran}
\bibliography{IEEEabrv,bib/bibtex,bib/non-paper}

\appendix
\chapter{Schematic of the Robot PCB}
\label{app:schematic_robot}

\begin{figure}[h!]
	\centering
	\includegraphics[width=0.9\textwidth]{pics/pcb_harvester_schematic.png}
	\caption{Energy harvester part of the schematic}
	\label{fig:pcb_harvester_schematic}
\end{figure}

\begin{figure}[h!]
	\centering
	\includegraphics[width=0.9\textwidth]{pics/pcb_sens_ctrl_schematic.png}
	\caption{Sensing and control part of the schematic}
	\label{fig:pcb_sens_ctrl_schematic}
\end{figure}

\end{document}

