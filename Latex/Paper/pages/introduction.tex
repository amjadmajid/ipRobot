\documentclass[../MainPaper.tex]{subfiles}
\begin{document}

\section{Introduction}

%- "Sectionize" the paper (sub-sections to ease the reading; sections you certainly need is "existing autonomous robots", "methods of energy harvesting for robotic and actuation properties", "types of actuation and their physical properties")
%- pose research statement in the introduction
%- focus on proper related work on autonomous robots (you already doing it, but looking at the list of papers you read the list is not complete here)

% Make people interested with the first sentence
%Intermittently powered robot research platforms, that rely on harvested power are currently non existent. 

Low cost robotic platforms have been developed to tackle a variety of challenges anonymously.
Miniature robots can be used for inspection in difficult to reach places, operating like mobile sensing units.
Hardware modularity is a way to make the robot adapt its resources to different environments and sensing operations.
By separating out power, computation, motor control and sensing a verity of capabilities can be tested \cite{RN13, RN2}.
Microrobots typically use infrared-based neighbor to neighbor distance sensing and communication \cite{RN3}.
While controlling a swarm or collective is mainly accomplished by means of active low power transceivers \cite{RN13, RN2}. 

%Single motor robot 1STAR complicated autonomy and everything is a turn

Choosing the right locomotion resource can depend on different factors, moving in the most energy efficient way on a particular surface is often the determining factor.
On a flat surface, robots commonly use a two-wheeled differential drive design to not only move but allow for steering as well \cite{RN13, RN2}. 
In other designs overall cost is a decisive factor, Kilobot uses two vibrating motors for locomotion.
When the motors are activated the centripetal forces will generate a forward movement \cite{RN3}.
The GRITSBot does not use conventional DC motors, requiring encoders to estimate their speed. 
Instead by using stepper motors the speed can be set by changing the delay between steps. 
Estimating it's position therefore is reduced to simply counting steps \cite{RN2}. 

%Why new robotic platform is designed to research intermittently powered

Robots still rely on batteries as a source of power, since electric motors consume considerably more energy compared to computation and sensing.
Lithium-ion batteries have a high energy density but still limit the operation time of the robot.
When the voltage of the battery drops below a certain level, indicating that the battery is almost empty, the robot needs to move to a charging station before it runs out of energy.
Replacement of the battery with a energy harvesting system would make the robot energy-autonomous. 
This research will explore the feasibility of a battery-less robot, taking into account the intermittently powered nature of it.
Intermittently powering robots creates new challenges in control, navigation and collaboration.

%Challenges in porting algorithms for localization

%Alternative energy source required since rf is not efficient 
%Especially when trying to combine communication with energy harvesting
%Compare sources
%Combining sources to improve operational time difficult to do


%%% OLD %%%

%Wireless Sensor Networks (WSN) are commonly used measure verity of different environmental conditions.
%The conditions measured can range from temperature, light humidity, to more complex sensing which requires additional computation.
%%In resent years the devices used for this telemetry have become low cost and battery less.
%Fully programmable platforms have been developed to exploring the combination of sensing and computation, while operating battery-less by making use of energy harvesting \cite{RN9}.
%The energy collected from RF signals is very minimal and decreases with the distance of the device to the source.
%RF energy can be rather intermittent in terms of availability, it requires the source to transmit.
%Other sources available for exploration are often limited by the application. Secondly, most sources can be scarce or completely absent during prolonged time intervals of the day as well \cite{RN15}.

%To allow more complex sensing, the energy budged needs to be evaluated carefully.
%Harvesting more energy increases the charge time but also increases the operation time.
%Most applications are not programmed to operate intermittently and therefore require enough energy to finish a single sensing operation. 
%To store the energy an appropriate size storage capacitor needs to be selected \cite{RN10}.
%A way to reduce the required energy is to communicate without active radios, but instead use an existing electromagnetic wave produced by a remote emitter. 
%Backscatter communication uses the wave and creates a signal by modulating the impedance of the antenna, causing a change in the amount of energy reflected back to the emitter \cite{RN9, RN15, RN10}.
%Even though WSN make it possible to cover big areas for sensing they are physically bound to a fixed location.
%Which in turn makes them "static" and require human intervention to make them cover new areas or change their location.

\end{document}
	
